\documentclass{physlab}

\begin{document}
\include{cover}

\paragraph{Цель работы:} При помощи модели абсолютно черного тела (АЧТ) провести измерения температуры оптическоим пиррометром с исчезающей нитью и термопарой. Исследовать излучение накаленных тел с различной испускательной способностью. Определить постоянные Планка и Стефана-Больцмана.

\section*{Теоритический аспект}
\hspace{7mm} Для измерения температуры разогретых тел, удаленных от наблюдателя, применяют методы оптической пирометрии, основанные на использовании зависимости испускательной способности исследуемого тела от температуры. 
Различают три температуры, функционально связанные с истинной термодинамической температурой и излучательной 
способностью тела: радиационную $T_{рад}$, цветовую $Т_{цв}$ и яркостную $T_{ярк}$.

\hspace{7mm} Под радиационной (энергетической) температурой понимают температуру абсолютно черного тела, при которой его интегральная испускательная способность одинакова с интегральной испускательной способностью исследуемого тела.

\hspace{7mm} Под цветовой температурой исследуемого тела понимают температуру абсолютно черного тела, при которой отношение их спектральных испускательных способностей для двух заданных длин волн одинаково.

\hspace{7mm} Под яркостной температурой понимают температуру абсолютно черного тела, при которой его спектральная испускательная способность равна спектральной испускательной способности исследуемого тела при той же длинне волны. Именно эту температуру мы будем измерять в данной работе. 

\hspace{7mm} Измерение яркостной температуры раскаленного тела производится при помощи оптического пирометра с исчезающей нитью, основанного на визуальном сравнении яркости раскаленной нити с яркостью изображения исследуемого тела. Равенство видимых яркостей, наблюдаемых через монохроматический светофильтр ($\lambda$ = 6500 $\angstrom$), фиксируется по исчезновению изображения нити на фоне раскаленного тела. Яркостный метод измерения температуры основан, в соответствии с формулой Планка, на зависимости испускательной способности абсолютно черного тела от температуры и длины волны.

\hspace{7mm} Оптический пирометр представляет собой зрительную трубу, внутри которой имеется накаливаемая нить, расположенная в плоскости изображения исследуемого раскаленного тела, а также темно-красный светофильтр ($\lambda$ = 6500$\angstrom$). Через окуляр одновременно наблюдается изображение исследуемого тела и раскаленной нити.

\hspace{7mm} Если в том узком спектральном интервале, который пропускается светофильтром, яркость нити меньше яркости раскаленного тела, то нить видится темной полоской на светлом фоне, и наоборот. При совпадении яркостей нить перестает быть видимой на фоне изображения раскаленного тела. Регулировка яркости нити осуществляется изменением тока, протекающего через нее.

\hspace{7mm} Шкалу прибора, измеряющего ток через нить, предварительно градуируют по абсолютно черному телу, термодинамическую температуру которого измеряют с помощью термопары. Если тело, температуру которого определяют, излучает как абсолютно черное тело, то мы можем с помощью пирометра найти его температуру. Если же тело излучает иначе, то определенное значение температуры является яркостной температурой.Яркостная температура тела всегда ниже его термодинамической температуры. Это связано с тем, что любое нечерное тело излучает меньше, чем абсолютно черное тело при той же температуре. Чтобы получить величину термодинамической температуры тела, надо вводить дополнительные поправки, которые определяются для каждого материала экспериментально.

\begin{figure}[H]
\centering
    \includegraphics[width=0.6\linewidth]{01}
\caption{График зависимости T = f($T_{\text{ярк}}$) для вольфрама}
\label{volf}
\end{figure}

\hspace{7mm} В данной работе используется оптический пирометр с исчезающей нитью, проградуированный при изготовлении по абсолютно черному телу, так что его цифровое табло во время измерения высвечивает значение температуры накаленного тела в градусах Цельсия. Вначале с помощью модели абсолютно черного тела проверяется правильность работы пирометра, а затем с его помощью исследуется излучение различных материалов и вольфрамовой нити накаливания. Необходимая для обработки проводимых в данной работе измерений зависимость между яркостной и термодинамической температурами вольфрама приведена на рис.$\ref{graphTT}$.

\begin{wrapfigure}{R}{0.5\linewidth}
	\caption{Таблица 1. Поправочные коэффициенты излучения для вольфрама} 
	\centering
    \includegraphics[width=0.9\linewidth]{02}
\label{tabl}
\end{wrapfigure}

\hspace{7mm} По результатам измерений мощности излучения вольфрамовой нити можно судить о справедливости закона Стефана-Больцмана. Для этого следует мощность, потребляемую питью, приравнять к излучаемому ею за единицу времени количеству энергии. Если бы нить излучала как абсолютно черное тело, то баланс потребляемой и излучаемой энергии определялся бы соотношением:

\begin{equation}
	W = \sigma S(T^4 - T_0^4) \text{,}
	\label{firstW}
\end{equation}


где W — потребляемая нитью электрическая мощность, S — площадь излучающей поверхности нити,  Т — температура нити, $T_0$ — температура окружающей среды. Однако вольфрамовая нить излучает как нечерное тело. Среди нечерных тел выделяются так называемые серые тела, для которых характер распределения излучения совершенно подобен спектру абсолютно черного тела, но излучение ослаблено по сравнению с ним в $\varepsilon_T$ раз для любой длины волны при данной температуре тела Т.

\hspace{7mm} Если предположить, что нить излучает как серое тело, то выражение ($\ref{firstW}$) можно записать в виде:

\begin{equation}
	W = \varepsilon_T \sigma ST^4 \text{,}
	\label{secondW}
\end{equation}

где мы учли, что реально температура вольфрама намного выше температуры окружающей среды. Значения коэффициента излучения $\varepsilon_T$ при различных температурах приведены в табл. 1. на рисунке $\ref{tabl}$

\begin{wrapfigure}{L}{0.7\linewidth}
	\centering
    \includegraphics[width=0.9\linewidth]{03}
    \caption{Распределение энергии в спектре издучения: \\ 1 - АЧТ \\ 2 - Вольфрам \\ Температура = 2450 К}
\label{energy}
\end{wrapfigure}

\hspace{7mm} Измерив температуру вольфрамовой нити в зависимости от подводимой мощности, можно убедиться в справедливости закона Стефана—Больцмана применительно к серому телу (в данном случае к вольфраму). Для этого нужно построить график зависимости W(T) в логарифмическом масштабе и по углу наклона определить показатель степени n исследуемой температурной зависимости. Понятно, что в пределах погрешности показатель степени должен быть близок к четырем.

\hspace{7mm} Из формулы ($\ref{secondW}$) можно определить также и величину постоянной $\sigma$ в законе Стефана—Больцмана. Некоторое отличие величин n и $\sigma$, полученных экспериментально, от теоретических значений может быть объяснено особенностью вольфрама, у которого наблюдается селективность излучения в коротковолновом диапазоне. Селективность излучения вольфрама становится особенно заметной при ярком накале, когда его температура составляет около 2400 К. Оказывается, что излучение в видимой области спектра существенно больше, чем это следует из распределения Планка, примененного к серому телу. Рис. $\ref{energy}$ иллюстрирует разницу между спектрами излучения абсолютно черного тела и вольфрама при температуре 2450 К.

\hspace{7mm} Именно поэтому вольфрам и выбран в качестве материала в лампах накаливания. При меньших температурах селективность излучения проявляется слабее, но при этом все большую роль играет теплоотвод от нити, что в свою очередь ведет к ошибке в определении величин n и $\sigma$.

\hspace{7mm} Проведя измерения в диапазоне температур от 800 до 1500 $^\circ C$, можно выяснить, в каком участке этого интервала температур воль-фрамовая нить лампы накаливания излучает почти как серое тело, т. е. величины п и сг соответствуют теоретическим значениям. 

\section*{Экспериментальная установка}

\hspace{7mm} Экспериментальная установка (рис. $\ref{scheme}$) состоит из оптического пирометра 9, модели абсолютно черного тела (АЧТ), трех исследуемых образцов (18, 19, 20), блока питания (1) и цифровых вольтметров В7-22А и В7-38 (14, 15, 16).

\begin{figure}[H]
\centering
    \includegraphics[width=0.8\linewidth]{04}
\caption{Схема экспериментальной установки: 1 — блок питания; 2 — тумблер включения питания пирометра и образцов; 3 — тумблер нагрева нити пирометра: «Быстро» — вверх, «Медленно» — вниз; 4 — кнопка «Нагрев нити»; 5 — кнопка «Охлаждение нити; 6 — тумблер переключения образцов; 7 — регулятор мощности нагрева образцов; 8 — окуляр пирометра; 9 — корпус пирометра; 10 — объектив пирометра; 11 — переключение диапазонов: 700 -1200$^{\circ}$ C — вниз, 1200—2000 $^{\circ}$ С — вверх; 12 — сектор красною светофильтра; 13 — регулировочный винт; 14 — вольтметр (напряжение на лампе накаливания); 15 — амперметр (ток через образцы); 16 — вольтметр в цепи термопары; 17 — модель АЧТ; 18 — трубка с кольцами из материалов с разной излучательной способностью; 19 — лампа накаливания; 20 — неоновая лампочка}
\label{scheme}
\end{figure}

\hspace{7mm}Пирометр 9 с исчезающей нитью включает в себя объектив 10, окуляр 8, монохроматический (красный) светофильтр 4, позволяюший рассматривать в лучах красного цвета (6500 $\angstrom$) нить пирометра на фоне изображения накаленного исследуемого тела. Перемещение светофильтра осуществляется сектором 12. Пирометр имеет два диапазона измерений: 700—1200 $^{\circ}$ С  и 1200—2000  $^{\circ}$ С. Переключение осуществляется введением серого светофильтра при помощи переключателя 11 «Включение». Регулировка накала нити пирометра выведена на лицевую панель блока питания.

\hspace{7mm}Модель АЧТ представляет собой керамическую трубку диаметром 3 мм и длиной 50 мм, закрытую с одного конца и окруженную для теплоизоляции внешним кожухом. Нагрев трубки осуществляется намотанной на ней нихромовой спиралью, питаемой от источника тока. Полость трубки и особенно ее дно излучают практически как абсолютно черное тело. Температура модели АЧТ измеряется хромель-алюмелевой термопарой, один спай которой вмонтирован в дно трубки, а другой находится при комнатной температуре на клемме цифрового вольтметра B7-38, измеряющего ЭДС термопары.

\hspace{7mm} В работе исследуются три образца. Один образец выполнен в виде керамической трубки с набором колец из различных материалов, нагреваемой изнутри нихромовой спиралью. Материалы колец имеют различную испускательную способность. Спираль подключается к источнику питания 1 с помощью переключателя 6 (положение 2) и может нагревать трубку до температуры около 1100$^\circ$ C. Термодинамическая температура колец практически одинакова и равна температуре трубки.

\hspace{7mm} Другой исследуемый образец — вольфрамовая нить электрической лампочки. Она питается от источника 1, когда переключатель 6 находится в положении 3. Сила тока через вольфрамовую нить измеряется с помощью прибора В7-22А (15). Падение напряжения на самой нити измеряется непосредственно вольтметром В7- 22А (16). Таким образом, зная показания обоих приборов, можно определить мощность, потребляемую нитью лампочки.

\hspace{7mm} Источник питания 1, используемый в работе, снабжен устройством, отключающим в случае перегрузки прибор от потребителя. Если это произойдет, то надо отключить питание прибора от сети 220 В и уменьшить напряжение на его выходе, а затем повторно включить источник питания.

\newpage

\section*{Ход работы}

\subsection*{Изучеие работы оптического пирометра}
\begin{enumerate}

\item C помощью пирометра измерим температуру модели АЧТ.

В данном пункте мы добивались исчезновения нити на фоне вечения дна модели АЧТ подбираю температуру двумя способами: снизу и сверху.

Полученные данные занемем в таблицу

\begin{table}[H]
\centering
\caption{Температура АЧТ, измеренная пирометром}
\begin{tabular}{|p{4cm}|p{3cm}|p{4cm}|p{3cm}|}
\hline
Значение $T_\text{АЧТ}$ подобранное снизу, $^\circ C$ & Показания термопары, мВ & Значение $T_\text{АЧТ}$ подобранное свеху, $^\circ C$& Показания термопары, мВ \\ \hline
1083                                             & 40,84                   & 1086                                 & 41,07                   \\ \hline
1090                                             & 41,17                   & 1089                                 & 41,13                   \\ \hline
1086                                             & 41,03                   & 1099                                 & 40,99                   \\ \hline
\end{tabular}
\end{table}

\hspace{7mm}Также из полученных данных для измерения одной и той же температуры можно заключить, что ошика в определении температуры $\Delta_T$ = 16$^\circ$С 


\item До начала эксперимента нами было снято значение напряжения термопары для случая, когда АЧТ был в ненагретом состоянии. 

\[U_\text{термопары при нулевом питании АЧТ} = -0,06\text{мВ}  \]

Из устройства термопары полученные данные будут учитываться следующим образом:
 
\[U_\text{действительное} = U_\text{термопары} - U_\text{термопары при нулевом питании АЧТ}\text{,} \]

Тогда для расчета значения температуры с помощью показаний термопары, мы получаем формулу:

\[T_\text{изм.} = [(U_\text{термопары} + 0,06)\cdot \alpha + 24]^\circ C \text{,}\]

где $\alpha$ - величина, полученная из граффика в описании лабораторного пратикума и равная:

\[\alpha = \frac{1000}{41}\]

\item Таким образом, собранные нами данные перезапишутся следующим образом.

\begin{table}[H]
\centering
\caption{Температура АЧТ, измеренная пирометром и термопарой}
\begin{tabular}{|p{4cm}|p{3cm}|p{4cm}|p{3cm}|}
\hline
Значение $T_\text{АЧТ}$ подобранное снизу, $^\circ C$ & Показания термопары, $^\circ C$ & Значение $T_\text{АЧТ}$ подобранное свеху, $^\circ C$& Показания термопары, $^\circ C$\\ \hline
1083                                             & 1022                   & 1086                                 & 1027                  \\ \hline
1090                                             & 1030                   & 1089                                 & 1029                   \\ \hline
1086                                             & 1026                   & 1099                                 & 1026                  \\ \hline
\end{tabular}
\end{table}

\hspace{7mm} Таким образом по собранным данным можно установить, что в статистическом иследовании ошибка между определением температуры с помощью пирометра и термопары не превышает 7\% (max $\delta$ = 6,6\%)   

\end{enumerate}

\subsection*{Измрение яркостной температуры накаленных тел}

\hspace{7mm} К сожаление в условиях проведения эксперемента наблюдать данное явление не представилось возможным.

\subsection*{Проверка законов Стефана-Больц}

\begin{enumerate}

\item Постепенно увеличивая при помощи ручки 7 накал нити лампы, начиная со слабого темно-красного накала 
($\sim$900$^\circ C$) вплоть до 1900 $^\circ C$, измерим пирометром яркостную температуру нити через каждые 100 $^\circ C$. При каждом измерении температуры запишем также величину тока и падения напряжения на нити лампы.

(Замечание: При измерении температуры свыше 1400$^\circ C$ необходимо переме-стить переключатель диапазонов 11 вверх — на 1200—2000 $^\circ C$.)

Собранные данные запишем в таблицу:

\begin{table}[H]
\centering
\caption{Данны для проверки закона Cтефана-Больцмана}
\begin{tabular}{|p{4cm}|p{4cm}|p{4cm}|}
\hline
$T_\text{пирометра}$, $^\circ C$ & I, А  & U, В \\ \hline
900                          & 0,5   & 1,95 \\ \hline
1000                         & 0,504 & 1,98 \\ \hline
1100                         & 0,536 & 2,27 \\ \hline
1200                         & 0,714 & 4,09 \\ \hline
1300                         & 0,722 & 4,2  \\ \hline
1400                         & 0,802 & 5,14 \\ \hline
1500                         & 0,842 & 5,62 \\ \hline
1600                         & 0,858 & 5,84 \\ \hline
1700                         & 0,899 & 6,39 \\ \hline
1800                         & 0,967 & 7,34 \\ \hline
1900                         & 1,079 & 9,02 \\ \hline
\end{tabular}
\end{table}

\item Для каждого значения измеренной яркостной температуры найдем термодинамическую темпеартуру вольфрамовой нити лампы, пользуясь графиком $\ref{graphTT}$, где T - абсолютная температура.

Также вычислим для каждого значения термодинамической температуры мощность, потребляемую нитью лампы.

Полученне значения занесем в таблицу:


\begin{table}[H]
\caption{Таблица для построения графика W = f(T)}
\label{tabl(W(T))}
\resizebox{\textwidth}{!}{
\begin{tabular}{|c|c|c|c|c|c|c|c|c|c|c|c|}
\hline
$T_\text{абсолютная}$, K & 1225	& 1327	& 1429	& 1531	& 1633	& 1735	& 1837	& 1939	& 2041	& 2143	& 2245
  \\ \hline
W(мощность), Вт               & 0,975 & 0,99792 & 1,21672 & 2,92026 & 3,0324 & 4,12228 & 4,73204 & 5,01072 & 5,74461 & 7,09778 & 9,73258 \\ \hline
\end{tabular}
}
\end{table} 

Представим результаты в виде графика:

\begin{figure}[H] 
\centering
\caption{Зависимость мощности, потребляемой нитью лампы от значения термодинамической температуры}
    \includegraphics[width=.9\linewidth]{graph1}
\label{W(T)}
\end{figure}

По даннным таблицы \ref{tabl(W(T))} потсроим граффик зависимости

\item Для проверки закона Стефана - Больцмана построим в логарифмическом масштабе график зависимости:

\[ W = \varepsilon_T B T^n \]
\[ \Downarrow \]
\[ ln(W) = ln(\varepsilon_T B) + n\cdot ln(T) \]

Тогда данные перезапишутся следующим образом:


\begin{table}[H]
\caption{Таблица для построения графика ln(W) = a + b$\cdot$f(T)}
\resizebox{\textwidth}{!}{
\begin{tabular}{|c|c|c|c|c|c|c|c|c|c|c|c|}
\hline
ln($T_\text{абсолютная}$) & 2,964  & 3,011  & 3,053 & 3,093 & 3,130 & 3,163 & 3,193 & 3,221 & 3,248 & 3,274 & 3,298 \\ \hline
ln(W)                 & -0,011 & -0,001 & 0,085 & 0,465 & 0,482 & 615   & 0,675 & 0,700 & 0,759 & 0,851 & 0,988 \\ \hline
\end{tabular}
}
\end{table}

В описании к лабораторной работе рекомендовано проводить исследование в области высоких температур. Таким образов в построении графика, последющей апроксимации его прямой и получении уравнения зависимости будут участвовать только последние 4 точки из таблицы:

\begin{figure}[H] 
\centering
\caption{Зависимость логарифма мощности, потребляемой нитью лампы от значения логарифма термодинамической температуры}
    \includegraphics[width=.9\linewidth]{graph2}
\label{lnW(lnT)}
\end{figure}

\hspace{7mm} Таким образом полученная звисимость апроксимируется по методу наименьших квадратов прямой:

\[ ln(W) = 4,49 \cdot ln(T) - 14,08 \] 

Найдем ошибку в определении значения n c помощью формулы для погрешности коэффициентов апроксимации МНК:

\[ \sigma_n = 1,35  \] 

\[ n = 4,49 \pm 1,35 \]

Таким образом с учетом погрешности мы получили значение n равное 4, что подверждает теорию изложенную в лабораторном практикуме.

\item Найдем величину постоянной Cтефана - Больцмана для каждого измеренного зачения T, превышающего 1700К по формуле:

\[ \sigma = \frac{W}{\varepsilon_{T} S T^4} \text{,}\]

где значение S = 0,36 см$^2$, значение $\varepsilon_T$ для каждого зачения температур возьмем из Таблицы 1 рис. $\ref{volf}$.

Запишем рассчитанные велечины в таблицу:

\begin{table}[H]
\caption{Значение величины постоянной Стефана - Больцмана}
\resizebox{\textwidth}{!}{
\begin{tabular}{|c|c|c|c|c|c|c|}
\hline
T, K                                                                                                                                                                   & 1735        & 1837        & 1939        & 2041        & 2143        & 2245        \\ \hline
$\varepsilon_T$                                                                                                                                                      & 0,441       & 0,439       & 0,437       & 0,435       & 0,433       & 0,431       \\ \hline
W, Вт                                                                                                                                                                    & 4,12228     & 4,73     & 5,01     & 5,74     & 7,10     & 9,73     \\ \hline
$\sigma$ $\cdot$ $10^{-12}$, Вт/см$^2$K$^4$  & 2,87 & 2,63 & 2,25 & 2,11 & 2,16 & 2,47 \\ \hline
\end{tabular}
}
\end{table}

\item Теперь зная велечину $\sigma$ мы можем оценить велицины постоянной Планка из формулы:

\[ h = \sqrt[3]{\frac{2\pi^5 k_\text{Б}^4}{15 \sigma c^2}} \]

Полученные значения для каждой $\sigma$ выпишем в таблицу:


\begin{table}[H]
\centering
\begin{tabular}{|c|c|c|c|c|c|c|}
\hline
h $\cdot$ 10$^{-34}$, Дж$\cdot$c	& 8,31 & 8,56 &	9,01 &	9,20 & 9,14 &	8,74 \\ \hline
$\sigma$ $\cdot$ $10^{-12}$, Вт/см$^2$K$^4$  &	2,87 & 2,63 & 2,25 & 2,11 & 2,16 &	2,47 \\ \hline
\end{tabular}
\end{table}

\item оценим точность с которой мы получили каждое из значений $\sigma$ и h по следующим формулам:

\[ \Delta_\sigma = 4 \Delta_T \cdot \frac{W}{\varepsilon_T S T^5} \]

\begin{equation}
\Delta_\sigma = \frac{4\Delta_T}{T} \cdot \sigma 
\label{deltasig}
\end{equation}

Аналогично:

\begin{equation}
\Delta_\sigma = \frac{1}{3} \cdot \frac{\Delta\sigma}{\sigma} \cdot h
\label{deltah}
\end{equation}

Выпишим по данным формулам ошибку для каждого измеренного h и $\sigma$ и занесем в таблицу:

\begin{table}[H]
\centering
\begin{tabular}{|c|c|c|c|c|c|c|}
\hline
h	& 8,31	        & 8,56	        & 9,01	        & 9,2	        & 9,14	       &  8,74        \\ \hline
$\Delta$h	     & 0,102	 & 0,099	 & 0,099	 & 0,096	 & 0,091	 & 0,083  \\ \hline
$\sigma$	       &2,87	        &2,63	        &2,25	        & 2,11	        & 2,16	        & 2,47        \\ \hline
$\Delta \sigma$	 & 0,106	 & 0,092	 & 0,074	 & 0,066	& 0,065	& 0,070 \\ \hline
\end{tabular}
\end{table} 

\hspace{7mm} К сожаления ни одно из полученный значений не равно табличныым даже с учетом погрешности. Ведь средняя погрешность не превышает проводимого эксперимента не превышает 5\%, когда значения разнять с табличными на величины порядка 30\%. Это говорит о том, что в проведенном опыте имеет место какая-то систематическая ошибка. Дальнешее расмотрение полученных величин не имеет смысла.

\end{enumerate}

\subsection{Измерении "яркостной температуры" неоновой лампочки} 

\hspace{7mm} При максимальной подаче напряжения на ноновую лампочку показания пирометра составили 925$^\circ$С. Однако, дотронувшись до лампочки, мы, превозмагая страх, уедились  в том, что термоденамическая температура не совпадает с измеренной.

\section*{Вывод}

\hspace{7mm}Проделав экспериименты представленные в лабнике, мы убедились в достоверности того факта, что излучение нагретых тел пропорционально четвертой степени их температуры. Однако полученные значения константных величин не совпали с табличными. Причиной данных расхождений могут служить следующие факторы: 1) Не учтена ошибка субъективного восприятия. 2)У вольфрама максимум плотности энергии излучения смещён по длине волны, поэтому при снятии значения со светофильтра возможно появление систематическая ошибки. Что мы и наблюдали в эксперименте, ведь все полученные значения систематически отклонялись от табличного на приблезительно одну величину. 

\hspace{7mm} Что касается несоответсвия яркостной и термодинамической температур неона, я считаю: Неон - прозрачный газ, следовательно его коэффициент отражения равен нулю, поэтому из закона Стефана-Больцмана следовало бы, что неон вообще не должен излучать(если тело не является АЧТ или серым). Но неон излучает, так как не описывается моделью АЧТ и серого тела, которая предполагает, что излучение порождается только лишь преобразованием термодинамической температуры тела в излучение, так как неон может преобразовывать в излучение не только термодинамическую температуру, но и энергию возбужденного состояния атомов.


\end{document}
