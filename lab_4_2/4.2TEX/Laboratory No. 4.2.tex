\documentclass{physlab}

\begin{document}
\include{cover}

\section{Теория $\beta$-распада}
Бета-распадом называется самопроизвольное превращение ядер, при котором их массовое число не изменяется, а заряд увеличивается или уменьшается на единицу. Бета-активные ядра встречаются во всей области значений массового числа $A$, начиная от единицы (свободный нейтрон) и кончая самыми тяжелыми ядрами. Период полураспада $\beta$-активных ядер изменяется от ничтожных долей секунды до $10^{18}$ лет. Выделяющаяся при единичном акте $\beta$-распада энергия варьируется от 18 кэВ (для распада трития ${}_1^3H$) до 13,4 МэВ (для распада изотопа бора ${}_5^{12}B$).
В данной работе мы будем иметь дело с электронным распадом 
\begin{equation*}
{}_Z^AX \longrightarrow {}_{Z+1}^AX + e^{-} + \tilde \nu,
\end{equation*}
при котором кроме электрона испускается антинейтрино. Освобождающаяся при $\beta$-распаде энергия делится между электроном, антинейтрино и дочерним ядром, однако доля энергии, передаваемой ядру, исчезающе мала по сравнению с энергией, уносимой электроном и антинейтрино. Практически можно сказать, что эти две частицы делят между собой всю освобождающуюся энергию. Поэтому электроны могут иметь любое значение энергии -- от нулевой до некоторой максимальной, которая равна энергии, освобождающейся при $\beta$-распаде, и является важной физической величиной. 

Вероятность $dw$ того, что при распаде электрон вылетит с импульсом $d^3 \vec{p}$, а антинейтрино с импульсом в интервале $d^3 \vec{k}$, очевидно, пропорциональна произведению этих дифференциалов. Нужно также учесть закон сохранения энергии, согласно которому импульсы $\vec{p}$ и $\vec{k}$ электрона и антинейтрино связаны соотношением 
\begin{equation} \label{E_save}
E_e - E - ck = 0,
\end{equation}
где $E_e$ -- максимальная энергия электрона, кинетическая энергия электрона $E$ связана с его импульсом обычным релятивистским соотношением 
\begin{equation}
E = c \sqrt{p^2 + m^2 c^2} - m c^2, 
\end{equation}
а через $ck$ обозначена энергия антинейтрино с импульсом $k$. Условие (\ref{E_save}) можно учесть введением в выражение для $dw$ $\delta$-функции
\begin{equation}
\delta (E_e - E - ck), 
\end{equation}
по определению не равной нулю только при соблюдении условия (\ref{E_save}).

Таким образом, вероятность $dw$ может быть записана в виде
\begin{equation}
dw = D \delta (E_e - E - ck) \  d^3 \vec{p} \  d^3 \vec{k} =  D \delta (E_e - E - ck) \ p^2 dp \ k^2 dk \ d\Omega_e \ d\Omega_{\tilde \nu}, 
\end{equation}
где $D$ -- некоторый коэффициент пропорциональности, $d\Omega_e,\ d\Omega_{\tilde \nu}$ -- элементы телесных углов направлений вылета электрона и нейтрино. Вероятность $dw$ непосредственно связана с $\beta$-спектром, поскольку для очень большого числа $N_0$ распадов число $dN$ распадов с вылетом электроноа и антинейтрино с импульсами соответственно от $\vec{p}$ до $\vec{p} + d\vec{p}$ и от $\vec{k}$ до $\vec{k} + d\vec{k}$ определяется соотношением 
\begin{equation}
dN = N_0 dw.
\end{equation}
Коэффициент $D$ в формуле (\ref{E_save}) с хорошей точностью можно считать для рассматриваемых нами распадов константой.

Величину $dw$ можно проинтегрировать по всем углам и по абсолютному значению импульса нейтрино. Интегрирование по каждому телесному углу дает множитель $4 \pi$, а 
\begin{equation}
\int_{- \infty} ^{+ \infty} f(x) \delta (x) dx = f(0).
\end{equation}
Поэтому, после домножения $dw$ на $N_0$ имеем
\begin{equation} \label{p-dist}
dN = \frac{16 \pi^2 N_0}{c^2}Dp^2(E_e - E)^2 dp, 
\end{equation}
где $dN$ обозначает уже число электронов, вылетающих из ядра с импульсом, величина которого лежит между $p$  и $p + dp$.

Чтобы получить распределение электронов по энергиям, надо в (\ref{p-dist}) перейти от $dp$ к $dE$:
\begin{equation}
dE = \frac{c^2 p}{E + mc^2} dp,
\end{equation}
после чего выражающая форму $\beta$ спектра величина $N(E) = dN/dE$ приобретает вид
\begin{equation} \label{E-dist}
\frac{dN}{dE} = N_0 B c p (E + mc^2)(E_e - E)^2 = N_0 B \sqrt{E (E + 2mc^2)} (E_e - E)^2 (E + mc^2), 
\end{equation}
где $B = (16 \pi^2 / c^4) D$. В нерелятивистском приближении, которое и имеет место в нашем случае, выражение (\ref{E-dist}) упрощается, и мы имеем
\begin{equation} \label{non-rel-E-dist}
\frac{dN}{dE} \approx \sqrt{E} (E_e - E)^2.
\end{equation}

\begin{wrapfigure}{r}{0.5\linewidth} \label{dist-pic} 
\vspace{-5ex}  
 \center{\includegraphics[width=.7\linewidth]{dist.png}}
\caption{Форма спектра $\beta$-частиц}
\end{wrapfigure}

Дочерние ядра, возникающие в результате $\beta$-распада, нередко оказываются возбужденными. Возбужденные ядра отдают свою энергию либо излучая $\gamma$-квант, либо передавая избыток энергии одному из электронов с внутренних оболочек атома. Излучаемые в таком процессе электроны имеют строго определенную энергию и называются конверсионными.

Итоговая форма спектра представлена на рис. \ref{dist-pic}. 
Спектр имеет вид широкого колокола с резким конверсионным максимумом, ширина которого определяется исключительно разрешающей способностью прибора. Кривая плавно отходит от нуля и столь же плавно, по параболе, касается оси абсцисс в области максимальной энергии электронов $E_e$. 

\section{Экспериментальная установка}

\begin{figure}[ht!] \label{scheme} 
 \center{\includegraphics[width=\linewidth]{scheme.png}}
\caption{Схема установки}
\end{figure}

Энергию $\beta$-частиц определяют с помощью $\beta$-спектрометров. В работе используются магнитный спектрометр с "короткой линзой". Его схема представлена на рис. \ref{scheme}. Электроны, испускаемые радиоактивным источником, попадают в магнитное поле катушки, ось которой параллельна оси $OZ$ (оси симметрии прибора). Траектории электронов в магнитном поле представляют собой сложные спирали, сходящиеся за катушкой в фокусе, расположенном на оси $OZ$. В фокусе установлен детектор электронов.

Как показывает расчет, для заряженных частиц тонкая катушка эквивалентна линзе. Ее фокусное расстояние $f$ зависит от импульса электронов $p_e$ и от индукции магнитного поля линзы (т.е. от силы тока $I$, протекающего через катушку) следующим образом:
\begin{equation} \label{focus}
\frac{1}{f} \propto \frac{I^2}{p_e^2}
\end{equation}

При заданной силе тока на входное окно счетчика фокусируются электроны с определенным импульсом. Электроны, обладающие другими значениями импульса, при этом не сфокусированы и в основном проходят мимо окна. При изменении тока в катушке на счетчик последовательно фокусируются электроны с разными импульсами. Так как геометрия прибора в течение опыта остается неизменной, импульс сфокусированных электронов пропорционален величине тока $I$:
\begin{equation}
p_e = kI.
\end{equation}

Рассмотрим связь между числом частиц, регистрируемых установкой, и функцией $W(p_e) = dW_e/dp_e$, определяемой формулой (\ref{non-rel-E-dist}). Как легко понять, 
\begin{equation}
N(p_e) \simeq W(p_e) \Delta p_e,
\end{equation}
где $\Delta p_e$~---~разрешающая способность спектрометра. Формула (\ref{focus}) показывает, что при заданном токе фокусное расстояние магнитной линзы зависит от импульса частиц. Мимо счетчика проходят частицы, для которых фокусное расстояние линзы слишком сильно отличается от нужного, т.е. при недопустимо больших $\Delta f$. Дифференцируя формулу (\ref{focus}) при постоянном токе, найдем:
\begin{equation}
\Delta p_e = \frac{1}{2} \frac{\Delta f}{f} p_e.
\end{equation}
Таким образом, величина интервала $\Delta p_e$, регистрируемого спектрометром, пропорциональна величине импульса. Окончательно получаем:
\begin{equation}
N(p_e) = C W(p_e) p_e, 
\end{equation}
где $C$~---~некоторая константа.

\section*{Ход работы}

\subsection*{Измерение фона.}

\begin{enumerate}

\item Значения фона, при котором проводился эксперимент по измерению спектра, выпишем в таблицу:

\begin{table}[H]
\centering
\caption{Полученные значения фона}
\begin{tabular}{|l|c|c|c|}
\hline
$t_{\text{изм}}$ & 60    & 120    & 500   \\ \hline
$N_\text{ф}$   & 0,599 & 0,6499 & 0,592 \\ \hline
d$N_\text{ф}$  & 0,1   & 0,074  & 0,034 \\ \hline
\end{tabular}
\end{table}

\item Имея эти данные мы можем доплнить нашу таблицу путем комбинации данных значений времени и количества частиц. Сделаем это для того, чтобы получить наиболее точное значение с наименьшей погрешностью имерения. 

\begin{table}[H]
\centering
\caption{Расширение таблицы со значением фона за счет имеющаихся значений:}
\begin{tabular}{|l|c|c|c|c|c|}
\hline
$t_{\text{изм}}$ & 60    & 120    & 180   & 500   & 680   \\ \hline
$N_\text{ф}$     & 0,599 & 0,6499 & 0,633 & 0,592 & 0,603 \\ \hline
d$N_\text{ф}$    & 0,1   & 0,074  & 0,059 & 0,034 & 0,03  \\ \hline
\end{tabular}
\end{table}

\item Таким образом, наиболее достоверным значеним фона для даннойго опыта является:

\[ N_\text{фон} = 0,603 \pm 0,03 \left[ \dfrac{\text{частиц}}{\text{c}} \right]\]

\item Стоит отметить, что фон прибора обусловен главным образом $\gamma$-квантами и электронами, рассеяными от стенок $\beta$-спектрометра.

\end{enumerate}

\subsection*{Собранные в ходе эксперимента значения}

\hspace{7mm} После проведения эксеримента на экране компьютера мы наблюдали следующую картину:

\begin{figure}[H]
\centering
\includegraphics[width=\linewidth]{data}
\caption{Cобранные данные}
\end{figure}


\subsection*{Нахождение колибровачной константы k}

\begin{enumerate}

\item При изменении тока в катушке на счетчик последовательно фокусируются электроны с разными импуьсами. Так как геометрия прибора в течении всего опыта остается неизменной , импульс сфокусированных электронов пропорционален величине тока I:

\[p_e = kI\]

\item Таким образом построив зависимость вида $N-N_\text{фон}$ от I надем конверсионное значение тока.


\begin{table}[H]
\caption{Собранные данные для построения зависимости N(I)} 
\resizebox{\textwidth}{!}{
\begin{tabular}{|l|c|c|c|c|c|c|c|c|c|c|c|c|c|c|c|c|}
\hline
№                     & 1     & 2     & 3     & 4     & 5     & 6     & 7     & 8     & 9     & 10    & 11    & 12     & 13     & 14     & 15     & 16    \\ \hline
I, А                  & 0     & 0,2   & 0,4   & 0,6   & 0,8   & 1     & 1,2   & 1,4   & 1,6   & 1,8   & 2     & 2,2    & 2,4    & 2,6    & 2,8    & 3     \\ \hline
N, 1/с                & 0,72  & 0,7   & 0,787 & 0,825 & 0,737 & 1,237 & 1,912 & 3,786 & 5,411 & 7,122 & 9,509 & 11,021 & 12,158 & 11,171 & 10,984 & 9,647 \\ \hline
N - $N_\text{ф}$, 1/с & 0,117 & 0,097 & 0,184 & 0,222 & 0,134 & 0,634 & 1,309 & 3,183 & 4,808 & 6,519 & 8,906 & 10,418 & 11,555 & 10,568 & 10,381 & 9,044 \\ \hline
\end{tabular}
}
\end{table}

\begin{table}[H]
\resizebox{\textwidth}{!}{
\begin{tabular}{|l|c|c|c|c|c|c|c|c|c|c|c|c|c|c|c|c|}
\hline
№                     & 16    & 17    & 18    & 19    & 20    & 21    & 22    & 23     & 24     & 25     & 26     & 27     & 28    & 29    & 30     & 31     \\ \hline
I, А                  & 3     & 3,2   & 3,4   & 3,6   & 3,8   & 4     & 4,1   & 4,2    & 4,25   & 4,3    & 4,35   & 4,37   & 4,4   & 4,6   & 4,8    & 5      \\ \hline
N, 1/с                & 9,647 & 6,66  & 4,798 & 2,399 & 2,124 & 1,649 & 7,785 & 12,433 & 13,06  & 13,745 & 13,793 & 12,211 & 9,197 & 1,612 & 0,55   & 0,512  \\ \hline
N - $N_\text{ф}$, 1/с & 9,044 & 6,057 & 4,195 & 1,796 & 1,521 & 1,046 & 7,182 & 11,83  & 12,457 & 13,142 & 13,19  & 11,608 & 8,594 & 1,009 & -0,053 & -0,091 \\ \hline
\end{tabular}
}
\end{table}

\begin{figure}[H]
\centering
\includegraphics[width=\linewidth]{N(i)}
\caption{$\beta$ - спектр с учетом фона}
\end{figure}



\item Значение пика по количеству частиц в секунду подбиралось таким образом, чтобы получиться как можно более точное значение. При отступе на $\pm$ 0,03 А значения получались меньше чем в пике. А учитывая тот факт, что показание амперметра флуктуировали примерно на это же значение, ток I = 4,35 $\pm$ 0,03 А (как видно из таблицы) можно считать конверисонным. 

\item Таким образом, зная, что величина произведения импульса конверсионного электрона на скорость света равна 1013,5 кэВ, получим значение калибровачной константы:

\[ k = 233 \pm 0,3 \text{кэВ}\] 

\item Стоит отметить, что подобный метод поиска константы был выбран нами не случайно:
В следсвие того, что погрешности в пике очень велики, значение тока в нем искалось графическим методом. Использовать каадратичную зависимость не имеело смысла, так как парабола "утонула" бы в этих погрешностях. 

\end{enumerate}

\subsection{Построение графика Ферми - Кюри}

\begin{enumerate}

\item Используя значение калбировочной константы, полученной в предыдущем пункте, мы имеем возможность перейти от зависимости N(I) к зависимости N(p). Проделаем это.

\item Проведем необходимые расчеты p для каждого значения тока I:

\begin{table}[H]
\caption{преобразованные данные для построения зависимости N(p)} 
\resizebox{\textwidth}{!}{
\begin{tabular}{|l|c|c|c|c|c|c|c|c|c|c|c|c|c|c|c|c|}
\hline
№                     & 1     & 2     & 3     & 4      & 5      & 6      & 7      & 8      & 9      & 10     & 11     & 12     & 13     & 14     & 15     & 16     \\ \hline
N - $N_\text{ф}$, 1/с & 0,117 & 0,097 & 0,184 & 0,222  & 0,134  & 0,634  & 1,309  & 3,183  & 4,808  & 6,519  & 8,906  & 10,418 & 11,555 & 10,568 & 10,381 & 9,044  \\ \hline
p/с                   & 0     & 46,60 & 93,20 & 139,79 & 186,39 & 232,99 & 279,59 & 326,18 & 372,78 & 419,38 & 465,98 & 512,57 & 559,17 & 605,77 & 652,37 & 698,97 \\ \hline
\end{tabular}
}
\end{table}

\begin{table}[H]
\resizebox{\textwidth}{!}{
\begin{tabular}{|l|c|c|c|c|c|c|c|c|c|c|c|c|c|c|c|c|}
\hline
№                     & 16     & 17     & 18     & 19     & 20     & 21     & 22     & 23     & 24     & 25      & 26      & 27      & 28      & 29      & 30      & 31      \\ \hline
N - $N_\text{ф}$, 1/с & 9,044  & 6,057  & 4,195  & 1,796  & 1,521  & 1,046  & 7,182  & 11,83  & 12,457 & 13,142  & 13,19   & 11,608  & 8,594   & 1,009   & -0,053  & -0,091  \\ \hline
p/c    & 698,97 & 745,56 & 792,16 & 838,76 & 885,36 & 931,95 & 955,25 & 978,55 & 990,20 & 1001,85 & 1013,50 & 1018,16 & 1025,15 & 1071,75 & 1118,34 & 1164,94 \\ \hline
\end{tabular}
}
\end{table}

\item Таким образом построенный по данным таблицы 4-5 граффик будет выглядить следубщим образом:

\begin{figure}[H]
\centering
\includegraphics[width=.9\linewidth]{N(p)}
\caption{Зависимость $\beta$ - спектра от импульса}
\end{figure}

\item Выполним переход от шкалы импульсов к шкале энергий, используя формулу:

\[T = \sqrt{p^2c^2 + m^2c^4} - mc^2\]

Получим следующие результаты:


\begin{table}[H]
\caption{преобразованные данные для построения зависимости N(E)} 
\resizebox{\textwidth}{!}{
\begin{tabular}{|l|c|c|c|c|c|c|c|c|c|c|c|c|c|c|c|c|}
\hline
$(pc)^2$, МэВ           & 0,00 & 2171,35 & 8685,38 & 19542,11 & 34741,53 & 54283,64 & 78168,45 & 106395,94 & 138966,13 & 175879,01 & 217134,58 & 262732,84 & 312673,79 & 366957,43 & 425583,77 & 488552,79 \\ \hline
№                     & 1,00 & 2,00    & 3,00    & 4,00     & 5,00     & 6,00     & 7,00     & 8,00      & 9,00      & 10,00     & 11,00     & 12,00     & 13,00     & 14,00     & 15,00     & 16,00     \\ \hline
N - $N_\text{ф}$, 1/с & 0,12 & 0,10    & 0,18    & 0,22     & 0,13     & 0,63     & 1,31     & 3,18      & 4,81      & 6,52      & 8,91      & 10,42     & 11,56     & 10,57     & 10,38     & 9,04      \\ \hline
E, кэВ                & 0,00 & 2,12    & 8,43    & 18,78    & 32,93    & 50,61    & 71,49    & 95,23     & 121,52    & 150,06    & 180,56    & 212,78    & 246,49    & 281,51    & 317,68    & 354,84    \\ \hline
\end{tabular}
}
\end{table}


\begin{table}[H]
\resizebox{\textwidth}{!}{
\begin{tabular}{|l|c|c|c|c|c|c|c|c|c|c|c|c|l|l|l|l|}
\hline
$(pc)^2$, МэВ           & 488552,7943 & 555864,5126 & 627518,9224 & 703516,0238 & 783855,8166 & 868538,301  & 912508,0525 & 957563,4768 & 980498,3163 & 1003704,574 & 1027182,25  & 1036649,317 & 1050931,344 & 1148641,903 & 1250695,153 & 1357091,095 \\ \hline
№                     & 16          & 17          & 18          & 19          & 20          & 21          & 22          & 23          & 24          & 25          & 26          & 27          & 28          & 29          & 30          & 31          \\ \hline
N - $N_\text{ф}$, 1/с & 9,044       & 6,057       & 4,195       & 1,796       & 1,521       & 1,046       & 7,182       & 11,83       & 12,457      & 13,142      & 13,19       & 11,608      & 8,594       & 1,009       & -0,053      & -0,091      \\ \hline
E, кэВ                & 354,8370483 & 392,8725091 & 431,6769979 & 471,1593678 & 511,2410756 & 551,8543178 & 572,3416139 & 592,9404317 & 603,2797298 & 613,6446435 & 624,0344708 & 628,197225  & 634,4485341 & 676,3343687 & 718,559333  & 761,0896569 \\ \hline
\end{tabular}
}
\end{table}

\begin{figure}[H]
\centering
\includegraphics[width=.9\linewidth]{N(e)}
\caption{Зависимость $\beta$ - спектра от энергии}
\end{figure}


\item Теперь преобразуем данные с целью построения графика Ферми-Кюри, откладывая по оси ординат $\dfrac{\sqrt{N(p)}}{p}$, а по оси абсцисс значение энергии:

\begin{figure}[H]
\centering
\includegraphics[width=.9\linewidth]{fermi.jpg}
\caption{Зависимость Ферми-Кюри}
\end{figure}

\item Средняя часть графика апроксимируется графиком прямой.
Путем экстраполяции получаем, что точка пересечения графика с осью обсцисс дает нам следующее значение:

\[ E_e \approx (550 \pm 15) \text{кэВ} \]

\end{enumerate}

\section*{Вывод}

В работе был исследован спектр $\beta$-распада ядра ${}^{137}Cs$. Полученная форма спектра совпадает с предсказанной теоретически. Более того полученное значение $E_e = 550 \pm 15 \text{ кэВ}$, согласующееся с табличным значением $514 \text{ кэВ}$ по порядку величины, но не совпдаает в пределах погрешности.

Причиной несовпадения данной величины с табичной может служить погрешность определния конверсионного пика: Как видно из графиков в ходе работы, усы погрешности в двух соседних значениях пересекались между собой, что давало пику некоторую подвижность. Именно это могло служить причиной различий поученного значения от табличного. Ведь в определении погрешности пика мы учитывали лишь погрешность флуктуирований тока, но не "свободу" пика. 

Более того, были неясности в определении фона, несмотря на попытки максимально точно его измерить, мы получал отрицательные значения пареметров N - $N_\text{ф}$. Что может служить причиной полагать, что фон менялся с течением времени.
\end{document}