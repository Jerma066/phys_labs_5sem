\documentclass{physlab}

\begin{document}
\include{cover}

\paragraph{Цель работы:} С помощью сцинтилляционного спектрометра исследуется энергетический спектр $\gamma$-квантов, рассеянных на графите. Определяется энергия рассеянных $\gamma$-квантов в зависимости от угла рассеяния, а также энергия покоя частиц, на которых происходит комптоновское рассеяние.

\section*{Теоритический аспект}
\hspace{7mm} Рассеяние $\gamma$-лучей в вществе относится к числу явлений, в которых особенно ясно проявляется двойственная природа излучения. Волновая теория хорошо объясняющая рассеяние длинноволнового излучения, испытывает трудности при описании рассеяния рентгеновских и $\gamma$-лучей. Эта теория, в частности, не может объяснить, почему в составе рассеяного излучения кроме исходной волны, с частотой $\omega_0$, появляется дополнительная длинноволная компонента, отсутсвующая в спектре первичного излучения.

\hspace{7mm} Появление этой компаненты легко объяснимо, если считать, что $\gamma$-излучение представляет с собой поток квантов(фтонов), имеющих энергию $\hbar\omega_0$ и импульс $p=\frac{\hbar\omega}{c}$. Эффект Комптона -- увеличение длины волны рассеянного излучения по сравнению с падающим -- интерпритируется как результат упругого соударения двух частиц: $\gamma$-кванта и сободного электрона.


\begin{wrapfigure}{l}{7cm} %{аргумент отвечает за местоположение картинки l - left, r - right}{место для размещения}
    \includegraphics[width = 7cm]{01}
    \caption{Векторная диаграмма \\ рассеяния}
\end{wrapfigure}
\hspace{7mm} Рассмотрим элементарную теорию эффекта Комптона. Пусть электрон до соударения покоился (его энерия равна энергии покоя $mc^2$), а $\gamma$-квант имел начальную энергию $\hbar\omega_0$ и импульс $\frac{\hbar\omega}{c}$. После соударения электрон приобретает энергию $\gamma mc^2$ и импульс $\gamma mv$, где $\gamma = (1-\beta^2)^{-\frac{1}{2}}$, $\beta = \frac{v}{c}$, а $\gamma$-квант рассеивается на некоторый угол $\theta$ по отношению к первоначальному направлению движения. Энергия и импульс $\gamma$-кванта становтся соотвественно равными $\hbar\omega_1$ и $\frac{\hbar\omega}{c}$ (рис. 1).

\hspace{7mm} Запишем для рассматриваемого процесса законы сохранения импульса (по обеим осям: вертикальной и горизонтальной) и энергии.+  


\[\text{ЗСЭ: } mc^2+\hbar \omega_0 = \gamma mc^2+\hbar \omega_1 \]
\[\text{ЗСИ по горизонтали: } \frac{\hbar\omega_0}{c} = \frac{\hbar\omega_1\cos\theta}{c}+\gamma mvcos\varphi \] 
\[\text{ЗСИ по вертикали: }\gamma mvsin\varphi = \frac{\hbar\omega_1}{c}sin\theta\]

\hspace{7mm} Решая совместно эти уравнения и переходя от частот $\omega_0$ и $\omega_1$ к длинам волн $\lambda_0$ и $\lambda_1$ нетрудно получить, что изменение длинны волны рассеянного излучения равно:

\begin{equation}
\Delta\lambda=\lambda_1-\lambda_0=\dfrac{h}{mc}(1-\cos\theta)=\Lambda_k(1-\cos\theta)
\label{deltlamb}
\end{equation}

где $\lambda_0$ и $\lambda_1$ - длины волн $\gamma$-кванта до и после рассеивания, а величина $\Lambda_k = \dfrac{h}{mc} = 2.42\cdot 10^{-10}$ \cm~---~комптоновская длина волны электрона. 

Из формулы (\ref{deltlamb}) следует, что комптоновское смещение не зависит ни от длинны волны первичного излучения, ни от рода вещества, в котором наблюдается рассеяние. В приведенном выводе электрон в атоме считается свободным. Для $\gamma$-квантов с энергией в несколько десятков, а тем более сотен килоэлектрон-вольт, связь электронов в атоме, действительно, мало существенна, так как энергия их связи не превосходит нескольких килоэлектрон-вольт, а для большинтсва электронов еще меньше.

\hspace{7mm}Эффект Комптона проявляется наиболее отчетливо при исспользовании в качестве рассеивателя легких элементов и при энергии $\gamma$-лучей порядка нескольких сотен килоэлектрон-вольт.
\hspace{7mm}Также стоит упомянуть, что кроме рассеивания $\gamma$-кванты испытывают в среде поглащение, вызываемое фотоэфектом и рождением электро-позитронных пар. Процесс рождения пар пороговый, он возможен лишь при энергии $\\gamma$-квантов более $2mc^2 = 1.02 \text{МэВ}$ и в рассматриваемом энергетическом диапозоне не происходит.

\hspace{7mm}Основной целью данной работы является проверка соотношения (\ref{deltlamb}). Применительно к условиям нашего опыта формулу (\ref{deltlamb}) следует преобразовать преходом от длин волн к энергии $\gamma$-квантов. Соотвествующее соотношение имеет вид:

\begin{equation}
\label{epstet}
\frac{1}{\varepsilon(\theta)}-\frac{1}{\varepsilon_0}=1-\cos\theta
\end{equation}
Где $\quad \varepsilon_0=\frac{E_0}{mc^2}$ -- выражение в единицах $mc^2$ энергия $\gamma$-квантов падающих на рассеиватель, $\varepsilon(\theta)$ -- выраженная в тех же единицах энергия квантов, импытавших комптновоское рассеяние на угол $\theta$, m-масса электрона.

\section*{Экспериментальная установка}

\begin{figure}[H]
\centering
    \includegraphics[width=0.6\linewidth]{02}
\caption{Блок~-~схема установки по изучению рассеяния $\gamma$-квантов: 1~-~источник излучения ($^{137}Cs$), 2~-~графитовая мишень, 3~-~фотоэлектронный умножитель (ФЭУ), 4~-~сцинтиллятор, 5~-~свинцовый коллиматор, 6~-~лимб}
\label{blockscheme}
\end{figure}

\hspace{7mm}Блок-схема установки изображен на рис. $\ref{blockscheme}$. Источником излучения служит $^{137}Cs$, испускающий $\gamma$-лучи с энергией 662 кэВ. Он помещен в тостостенный свинцовый контейнер с коллиматором. Сформированный коллиматором узкий пучок узкий пучок $\gamma$-квантов попадает на графитовую мишень 2 (цилиндром диаметром 40 мм и высотой 100 мм).

\hspace{7mm}Кванты испытыавшие комптоновское рассеяние в мишени, регистрируются сцинтелиционным счетчиком, который состоит из фотоэлектронного умножителя(ФЭУ) 3 и осцилятора 4. Сцинтелятором служит кристалл NaI(T1) цилиндрической формы диаметром 40 мм и и высотой 40 мм, его выходное окно находится в оптическом контакте с фотокатолом ФЭУ. Сигналы, возникающие на аноде ФЭУ подаются на ЭВМ для амплитудного анализа. Кристалл и ФЭУ расположены в светонепроницаемом блоке, укрепленном  на горизонтальной штанге. Штанга может вращаться относительно мишени, угол поворота отсчитывается по лимбу 6.  
\hspace{7mm} Головная часть сцинтилляционного блока закрыта свинцовым коллиматором 5, который формирует входной пучок и защищает детектор от постороннего излучения. Основной вклад в это излучение вносят $\gamma$-кванты, проходящие из источника 1 через 6-сантиметровые стенки защитного контейнера. Этот фон особенно замтен при исследовании комптоновского рассеяния на большие углы ($\sim120$), когда расстояние между детектором и источником уменьшается.

\begin{figure}[H]
\centering
    \includegraphics[width=0.4\linewidth]{03}
\caption{Блок~-~схема измерительного комплекса: Д~-~дисплей, ПР~-~принтер, ВСВ~-~высоковольтный выпрямитель, УА~-~усилитель~-~анализатор, КЛ~-~клавиатура}
\label{funcblockscheme}
\end{figure}  

\hspace{7mm} На рис. $\ref{funcblockscheme}$ представленна функциональная блок-схема измерительного комплекса, который состоит из ФЭУ, питаемого от высоковольтного выпрямителя ВСВ, обеспечивающего работу ФЭУ в спектрометрическом режиме, усилителя-анализатора УА, являющегося исходным интерфейсом ЭВМ, управлемоц клавиатуры КЛ. В ходе проведения эксперимента информация отражается на экране дисплея Д.

\section*{Ход работы}
\begin{enumerate}

\item Устанавливая сцинтилляционный счётчик под разными углами $\theta$ к первоначальному направлению полёта $\gamma$-квантов, и вводя значения этих углов в ЭВМ,снимем амплитудные спектры и определили положение фотопиков для каждого значения угла $\theta$.
	
На экране ЭВМ были получены следующие спектры:
	
\begin{figure}[H]
\caption{зависимость спектра на экране ЭВМ от $\theta$}
\begin{minipage}[h]{0.47\linewidth}
\center{\includegraphics[width=1\linewidth]{0}} $\theta$=$0^\circ$ \\
\end{minipage}
\hfill
\begin{minipage}[h]{0.47\linewidth}
\center{\includegraphics[width=1\linewidth]{10}}  $\theta$=$10^\circ$ \\
\end{minipage}
\vfill
\begin{minipage}[h]{0.47\linewidth}
\center{\includegraphics[width=1\linewidth]{20}}  $\theta$=$20^\circ$ \\
\end{minipage}
\hfill
\begin{minipage}[h]{0.47\linewidth}
\center{\includegraphics[width=1\linewidth]{30}}  $\theta$=$30^\circ$ \\
\end{minipage}
\vfill
\begin{minipage}[h]{0.47\linewidth}
\center{\includegraphics[width=1\linewidth]{40}}  $\theta$=$40^\circ$ \\
\end{minipage}
\hfill
\begin{minipage}[h]{0.47\linewidth}
\center{\includegraphics[width=1\linewidth]{50}}  $\theta$=$50^\circ$ \\
\end{minipage}
\vfill
\begin{minipage}[h]{0.47\linewidth}
\center{\includegraphics[width=1\linewidth]{60}}  $\theta$=$60^\circ$ \\
\end{minipage}
\hfill
\begin{minipage}[h]{0.47\linewidth}
\center{\includegraphics[width=1\linewidth]{70}}  $\theta$=$70^\circ$ \\
\end{minipage}
\end{figure}
	
\newpage
\begin{figure}[H]
\begin{minipage}[h]{0.47\linewidth}
\center{\includegraphics[width=1\linewidth]{80}} $\theta$=$80^\circ$ \\
\end{minipage}
\hfill
\begin{minipage}[h]{0.47\linewidth}
\center{\includegraphics[width=1\linewidth]{90}}  $\theta$=$90^\circ$ \\
\end{minipage}
\vfill
\begin{minipage}[h]{0.47\linewidth}
\center{\includegraphics[width=1\linewidth]{10}}  $\theta$=$100^\circ$ \\
\end{minipage}
\hfill
\begin{minipage}[h]{0.47\linewidth}
\center{\includegraphics[width=1\linewidth]{110}}  $\theta$=$110^\circ$ \\
\end{minipage}
\centering
\begin{minipage}[h]{0.47\linewidth}
\center{\includegraphics[width=1\linewidth]{120}}  $\theta$=$120^\circ$ \\
\end{minipage}
\end{figure}

\item Таким образом собранные нами данные запишем в таблицу:	

\begin{table}[H]
\caption{Собранные данные}
\resizebox{\textwidth}{!}{
\begin{tabular}{|c|c|c|c|c|c|c|c|c|c|c|c|c|c|}
\hline
$\theta^\circ$ & 0    & 10   & 20  & 30  & 40  & 50  & 60  & 70  & 80  & 90  & 100 & 110 & 120 \\ \hline
Канал $\equiv$ N($\theta$)      & 775  & 726  & 512 & 675 & 596 & 530 & 464 & 417 & 366 & 344 & 312 & 277 & 264 \\ \hline
$N_{max}$              & 1616 & 1172 & 87  & 197 & 134 & 112 & 134 & 96  & 93  & 135 & 137 & 149 & 174 \\ \hline
\end{tabular}
}
\label{table}
\end{table}

\newpage

\item Заменим в формуле (\ref{epstet}) энергию квантов, испытавших комптоновское рассеяние на угол $\theta$, номеро Канала N($\theta$), соответсвующее вершине фотопика при указанном угле $\theta$. Обозначая буквой A неизвестный коэффициент пропорциональности между $\epsilon(\theta)$ и N($\theta$), найдем:

\begin{equation}
\label{obr}
\dfrac{1}{N(\theta)}-\dfrac{1}{N(0)}= A(1-\cos\theta)
\end{equation}

\item Используя данные таблицы (\ref{table}) построим зависимость 1/N($\theta$) от 1-cos($\theta$)

К сожалению из рассмотрения придется убрать точку со значение $\theta$=$20^\circ$ так как она черезчур сильно отличается от значения апроксимирующей кривой. За погрешность измерения N($\theta$) будем брать 20 каналов. Относительная погрешность какой-либо величины не отличается от относительной погрешности обратной. Таким образом:

\begin{equation}
\label{pogr}
\sigma\left(\frac{1}{N(\theta)}\right) = \frac{20}{N(\theta)}\cdot\frac{1}{N(\theta)} 
\end{equation}

\begin{table}[H]
\caption{Данные для построения графика зависимости N($\theta$) от 1-cos($\theta$)}
\resizebox{\textwidth}{!}{
\begin{tabular}{|c|c|c|c|c|c|c|c|c|c|c|c|c|c|}
\hline
1-cos($\theta$) & 0           & 0,015      & 0,06        & 0,134       & 0,234       & 0,357       & 0,5         & 0,658       & 0,826      & 1           & 1,174       & 1,342       & 1,5         \\ \hline
1/N($\theta$)   & 0,001290 & 0,001378 & 0,001953 & 0,001481 & 0,001678 & 0,001887 & 0,002155 & 0,002398 & 0,002732 & 0,002907 & 0,003205 & 0,003610 & 0,003788 \\ \hline
$\sigma\left(\frac{1}{N(\theta)}\right)$ & 0,0000666 &	0,0000759 &	0,000153 &  	0,0000877 &	0,000112 & 0,000142 &	0,000185 & 0,000230 & 0,000298 & 0,000338 &	0,000411 & 0,000521 &	0,000574 \\ \hline
\end{tabular}
}
\end{table}

Построим граффик по методу наименьших квадратов:

\begin{figure} [H]
   \centering
   \caption{График зависимости $\dfrac{1}{N(\theta)}-\dfrac{1}{N(0)}= A(1-\cos\theta)$}
     \includegraphics[width=1\linewidth]{graph}
\end{figure}

\newpage

\item Оценим достоверность апроксимации методом $\chi^2$ по формуле:

\begin{equation}
\label{chi}
\chi^2 = \sum_{i=1}^{N}\left(\frac{y_{i, exp}-y_{i, theor}}{\sigma_i}\right)^2
\end{equation}
			
$\chi^2$ = 3,901592883 $\Rightarrow$ /из таблицы на странице 431 Лаб. прак./ Достоверность апроксимации 95%
		
Теперь найдем погрешность метода для коэффициента перед x - k и свободного \\ члена - b; Тогда:

\hspace{4cm}k = 0,0017 $\pm$ 0,00009315 $\Rightarrow$ $\varepsilon_k$ = 5,48\% 

\hspace{4cm}b = 0,0013 $\pm$ 0,00004659 $\Rightarrow$ $\varepsilon_b$ = 3,58\%

\item Согласно формуле (\ref{obr}) экспериментальные точки должны лежать на одной прямой. Пересечение этой прямой с осью координат определяется наилучшее значение $N_{\text{наил.};}$(0). Это значение учиывает не только непосредственно измеренную величины N(0), но и измерения сделанное под другим углами, а пересечение линии с прямой $\cos(\theta)$ = 0 позволяется найти  $N_{\text{наил.}}$(90). Сравенние результатов опыта с теоретическими оценками позволяет найти энергию покоя частиц, на которых происходит комптоновское рассяние (по нашему предположению электронов). Для этого снова обратимся к формуле (\ref{epstet}). Возвращаясь от переменной $\varepsilon$ к энергии E ($\varepsilon=\frac{E}{mc^2}$), мы получаем, что при $\theta$ = $90^\circ$ формула ($\ref{epstet}$) принимает вид:

\[ mc^2\left(\frac{1}{E(90)} - \frac{1}{E(0)}\right) = 1 \]

\hspace{8cm}$\Downarrow$ (или)
\begin{equation}
\label{mc2}
mc^2= E(0)\frac{E(90)}{E(0)-E(90)} \Leftrightarrow E_\gamma\cdot\dfrac{N(90)}{N(0)-N(90)} \text{,}
\end{equation}

где $E_\gamma$~---~энергия $\gamma$-лучей, рассеянных источником.

Таким образом из граффика:


\[	\frac{1}{N_{\text{наил.}}(0)} = 0,0013 \Rightarrow N_{\text{наил.}}(0) = 769,23	\]
\[	\frac{1}{N_{\text{наил.}}(90)} = 0,003 \Rightarrow N_{\text{наил.}}(90) = 333,33 \]
\[	E_{\gamma} = 662 \K\eV													\]			
\[	E_{\text{эксп.}} = mc^2 = 662 \K\eV\cdot\frac{333,33}{769,23 - 333,33}	\]
\[ \Downarrow\]
\[ E_{\text{эксп.}} \approx 506\K\eV											\]

\item Рассчитаем погрешности измерений:
        \[\dfrac{\sigma_{N(0)}}{N(0)}=\dfrac{\sigma_b}{b}\approx0,0358 \]
        \[\dfrac{\sigma_{N(90)}}{N(90)}=\sqrt{\left(\frac{\sigma_b}{b}\right)^2 + \left(\frac{\sigma_k}{k}\right)^2}=\sqrt{(0,0358)^2 + (0,0548)^2} \approx 0.0654 \]
		\[\sigma(E_{\text{эксп.}}) = E_{\gamma}\cdot\sqrt{\left(\sigma_{N(90)}\cdot\frac{N(0) - 2N(90)}{(N(0) - N(90))^2}\right)^2   +   \left(\sigma_{N(0)}\cdot\frac{N(90)}{(N(0) - N(90))^2}\right)^2 }\]
		\[ \Downarrow \]

\[\sigma(E_{\text{эксп.}}) = 32,87 \K\eV \]

\hspace{7mm} Таким образом величина энергии покоя частицы, на которой происходит компотоновское рассеяние первичных $\gamma$-квантов:

\[ \text{Е}_{\text{покоя}} = (506 \pm 33)\K\eV \] 

\end{enumerate}	

\section*{Вывод}

\hspace{7mm} В данной работе мы исследовали энергетический спектр $\gamma$-квантов, рассеянных на графите. Пронаблюдав зависимость спектров рассеивания на экране ЭВМ, мы попыались апроксимировать его линейной функциией. Как показал метод $\chi^2$, полученные в ходе эсперимента значения апроксимируются прямолинейным графиком вида y = 0,0017x + 0,0013 с достоверностью 95\%, что означает, справедливость формулы $\frac{1}{N(\theta)}$ = 0,0017$\cdot(1-\cos\theta) + 0,0013$, где N - номер канала в анализаторе, $\theta$ - угол рассеяния.  Использовав полученный результат, мы определили энергию покоя частицы, на которой происходит рассеяние: $E_{\text{эксп.}}=506\pm33 \; \K \eV$. Стоит отметить, что полученная величина оказалась равна энергии покоя электрона в пределах погрешностей, величина которой: $E_{\text{электр.}}\approx511 \; \K \eV$. Таким образом нам удалось подтвердить, что  рассеяние происходит на электронах. Относительная погрешность полученного значения энергии составляет $\approx$ $6,5\%$.

\newpage
Еще у нас есть это!
	
\begin{figure}[H]
\begin{minipage}[h]{0.47\linewidth}
\center{\includegraphics[width=1\linewidth]{0-40}} $\theta$=$0^\circ$ - $40^\circ$ \\
\end{minipage}
\hfill
\begin{minipage}[h]{0.47\linewidth}
\center{\includegraphics[width=1\linewidth]{50-90}}  $\theta$=$50^\circ$ - $90^\circ$ \\
\end{minipage}
\centering
\begin{minipage}[h]{0.47\linewidth}
\center{\includegraphics[width=1\linewidth]{100-120}}  $\theta$=$100^\circ$ - $120^\circ$ \\
\end{minipage}
\end{figure}

\end{document}
