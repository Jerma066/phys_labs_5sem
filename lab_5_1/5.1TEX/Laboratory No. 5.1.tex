\documentclass{physlab}

\begin{document}
\include{cover}

\paragraph{Цель работы:} С помощью сцинцтиляционного счётчика измеряются линейные коэффициенты ослабления потока $\gamma$-лучей  в свинце, железе и алюминии; по их величине определяется энергия $\gamma$-квантов.

\section*{Теоритический аспект}
Гамма-лучи возникают при переходе возбуждённых ядер в более низкое энергетическое состояние. Энергия $\gamma$-квантов обычно порядка ~$10\div1000$ кэВ. Заряд и масса $\gamma$-кванта равны нулю. Проходя через вещество, пучок $\gamma$-квантов ослабляется по закону:
\begin{equation}
    I = I_0e^{-\mu l}
\label{eq:coef}
\end{equation} 
	или
\begin{equation}
    I=I_0e^{-\mu' m_1},
\end{equation}
где $I, I_0$ - интенсивности прошедшего и падающего излучений, $l$~---~длина пути, пройденного  пучком $\gamma$-лучей, $m_1$~---~масса пройденного вещества на единицу площади, $\mu$ и $\mu'$~---~константы, зависящие от среды ($[\mu] = \text{см}^{-1}$, $[\mu'] = \text{см}^{2}/\text{г}$). $\mu'$, в отличие от $\mu$, не зависит от плотности среды. Ослабление потока $\gamma$-лучей в веществе связано с тремя эффектами: фотоэлектрическим поглощением, комптоновским рассеянием и генерацией электрон-позитронных пар.

\textbf{Фотоэлектричекое поглощение}

	При столкновении $\gamma$-квантов с электронами внутренних атомных оболочек может происходить поглощение квантов. Свободные (наружные) электроны не могут поглощать кванты.Вероятность $dP_\text{ф}$ фотоэлектрического поглощения $\gamma$-квантов: 
\[ dP_\text{ф}=\sigma_\text{ф} n_1 dl, \]
	где $dl$~---~длина пути, $n_1$~---~плотность внутренних  электронов, $\sigma_\text{ф}$~---~поперечное сечение фотоэлектрического поглощения.
\[ \mu_\text{ф}=\sigma_\text{ф}n_1, \]	
	$\mu_\text{ф}$~---~коэффициент поглощения для фотоэффекта $\mu$ из уравнения \eqref{eq:coef}.
	
	Фотоэффект является доминирующим механизмом поглощения $\gamma$-квантов при не очень высоких энергиях. Его вероятность зависит от энергии лучей и заряда  ядер.
\begin{figure}[H]
\centering
    \includegraphics[width=0.45\linewidth]{01}
\caption{Зависимость сечения фотоэффекта от энергии $\gamma$-квантов.}
\end{figure}

\textbf{Комптоновское рассеяние}

Комптоновское рассеяние~---~упругое столкновение $\gamma$-кванта с электроном. Оно может происходить на свободных/слабосвязанных электронах. Эффект Комптона становится существенным, когда энергия квантов становится много больше энергии связи электронов в атоме. В этом случае сечение комптон-эффекта:
\begin{equation}
    \sigma_K=\pi r^2 \dfrac{mc^2}{\hbar \omega}\left(ln\frac{2\hbar\omega}{mc^2}+\frac{1}{2}\right),
\end{equation}
где $r\simeq 2.8 \cdot 10^{-13}$ см~---~классический радиус электрона, $m$~---~его масса.
	
Эффект комптона приводит не к поглощению, а к рассеянию $\gamma$-квантов и уменьшению их энергии.

\textbf{Образование пар}

При энергиях $\gamma$-лучей больше 1.02 МэВ становится возможным поглощение лучей, связанное с образованием электрон-позитронных пар. Оно возникает в электрическом поле ядер. Вероятность этого процесса приблизительно пропорциональна $Z^2$.
	   
\textbf{Полный коэффициент ослабления потока $\gamma$-лучей}
	 
Полный коэффициент ослабления потока лучей равен сумме коэффициентов для трёх рассмотренных процессов. 
	
\begin{figure}
\centering
    \includegraphics[width=0.5\linewidth]{02}
\caption{Полные коэффициенты ослабления потока $\gamma$-лучей в алюминии, железе и свинце.}
\end{figure}
	
 	Полный коэффициент ослабления:
 	
 	\begin{equation}
 	\mu=\frac{1}{l}ln\frac{N_0}{N}
 	\end{equation}
	
	В работе определяются толщина образца $l$, число падающих частиц $N_0$ и число прошедших частиц $N$.
	
\section*{Экспериментальная установка}
	
\begin{figure} [h!]
	\centering
    \includegraphics[width=0.8\linewidth]{03}
	\caption{Блок-схема установки, используемой для измерения коэффициентов ослабления потока $\gamma$-лучей; Pb~---~свинцовый контейнер с коллиматорным каналом; П~---~набор поглотителей, ПП~---~пересчётный прибор; С~---~сцинтиллятор (кристалл $NaI(Tl)$); ВВ~---~высоковольтный выпрямитель, Ф~---~формирователь-выпрямитель; И~---~источник $\gamma$-лучей}
\end{figure}

\begin{figure} [h!]
	\centering
	\includegraphics[width=0.9\linewidth]{04}
	\caption{Схема рассеяния $\gamma$-квантов в поглотителе}
\end{figure}
			
	

\section*{Ход работы}

\subsection*{Вычисление фона.}

\hspace{7mm} При дальйнеших измерениях необходимо вычитать
фон, который обусловлен шумом ФЭУ и посторонними частицами:
космическим излучением, $\gamma$-квантами от соседних источников,
квантами, рассеянными на стенах комнаты и в стенках прибора, и т.д.

\begin{enumerate}

\item Закрыв коллиматор толстой свинцовой пробком, проведем несколько измерений фона за время t = 10 сек. и запишем их в таблицу 1:

\begin{table}[H]
\centering
\caption{Измерение фона за промежуток времени = 10 сек.}
\begin{tabular}{|l|c|c|c|c|c|}
\hline
№ измерения & 1   & 2   & 3   & 4   & 5   \\ \hline
N, частиц   & 106 & 110 & 123 & 140 & 100 \\ \hline
\end{tabular}
\end{table}

\item Фон и погрешность фона будем считать по формулам (\ref{fon}) и (\ref{pogfon}) соотвественно:

\begin{equation}
\label{fon}
N_\text{фон} = \frac{\sum_{i=1}^{n}N_i}{n}
\end{equation}

\begin{equation}
\label{pogfon}
\sigma_N = \frac{1}{\sqrt{n\cdot (n+1)}} \cdot \sqrt{\sum_{i=1}^{n} (N_\text{фон}-N_i)^2}
\end{equation}

\item Таким образом имеем следущие значения:

\[N_\text{фон}\approx 116\]
\[\sigma_N \approx 6 \]

\item Отсносительная ошибка измерения состовляет менее 6\%. А значит полученное значения фона достаточно достоверно, чтобы мы могли учитывать его для корректировки последущих измерений.

\end{enumerate}

\subsection*{Измерение в осутсвие поглотителя}

\hspace{7mm} Прроведем упомянутое измерение 3 раза и рассчитаем ошибку полученного значения по формулам ($\ref{fon}$) и ($\ref{pogfon}$):

\hspace{7mm} В отсутствие поглотителя число частиц попадающих в счетчик за t = 10 секeyд  составляет: 

\begin{table}[H]
\centering
\caption{Число частиц попадающих на счетчик в отсутвие поглотителя}
\begin{tabular}{|l|c|c|c|}
\hline
№ эксперимента & 1     & 2     & 3     \\ \hline
$N_0$, частиц      & 46584 & 48825 & 42831 \\ \hline
\end{tabular}
\end{table}

\[N_0 \approx 46080 \text{ частиц} \]
\[\sigma_N \approx 1237 \text{ частиц}\]

\subsection*{Исследование поглощений $\gamma-$ лучей в Алюминии}

\begin{enumerate}

\item Предварительно многократно, а имеено 10 раз, проведем измерние числа частиц попадающх в счетчик за 10 секунд при наличии одного поглотителя алюминия толщиной 2 сантиметра и запишем полученные данные в таблицу 3:

\begin{table}[H]
\centering
\caption{Количество частиц при одном поглатителе за 10 секунд}
\begin{tabular}{|l|l|l|l|l|l|l|l|l|l|l|}
\hline
N, частиц &	31235 &	31370 &	30584 &	30713 &	30781 &	30553 &	30661 &	30711 &	30310 &	30121 \\ \hline
t, сек &	10 & 10 & 10 & 10 &	10 & 10 & 10 & 10 &	10 & 10                                  \\ \hline
l, мм &	20 & 20 & 20 & 20 & 20 & 20 & 20 & 20 & 20 & 20                                   \\ \hline
\end{tabular}
\end{table}

\item Таким образом используя формулы (\ref{fon}) и (\ref{pogfon}) рассчитаем число частиц попавших в счетчик и сучайную ошибку проводимых экспериментов: 

\[N_\text{10сек}\approx 30704 \text{ частиц}\]
\[\sigma_N \approx 51 \text{ частиц} \] 

\item Отноcительная ошибка такого измерения составляет $\approx$ 0,17\%. Полученное значение ошибки заметно меньше ошибки соотвествующей пуасоновскому распределению, которая состовляет на данном измерении $\approx$ 0,5\% и будет тем больше, чем меньше зарегистрированное значение частиц. Таким образом приблизится к величине $\sigma_N$ можно было бы увеличив время регистрирования частиц (подобно тому, как это происходит в случае, когда мы увеличиваем каоличество экспериментов по измерению однй и той эе велчины с целью уменьшения случайно погрешности). К сожалению, такой шаг не был предпринят и теперь мы будем использовать следующую формулу для крестов погрешности:

\[\sigma_N \approx \sqrt{N_i} \text{ частиц} \] 

\item Проведем измерение числа частиц, попадающих в счетчик за время t = 10 секунд, при различных частицах поглотителя. Полученные данные выпишем в Таблицу 4 (k - колличество поглотителей, l - сумарная толщина в мм):

\begin{table}[H]
\caption{Поглотитель -- алюминий, t = 10 секунд.} 
\resizebox{\textwidth}{!}{
\begin{tabular}{|l|c|c|c|c|c|c|c|c|c|c|c|}
\hline
k, штук        & 1     & 2     & 3     & 4    & 5    & 6    & 7    & 8    & 9   & 10  & 11  \\ \hline
l, мм          & 20    & 40    & 60    & 80   & 100  & 120  & 140  & 160  & 180 & 200 & 220 \\ \hline
N, частиц      & 30704 & 18042 & 11295 & 7166 & 4472 & 2845 & 1878 & 1242 & 813 & 553 & 376 \\ \hline
N - $N_\text{фон}$, частиц & 30588 & 17926 & 11179 & 7050 & 4356 & 2729 & 1762 & 1126 & 697 & 437 & 260 \\ \hline
\end{tabular}
}
\end{table}

\item Выведем формулу для посчетов крестов погрешности в данном эксперименте; Для логарифма справедливо:

\[ \delta y = ln(x) - ln(x-\delta x) \]
\[ \downarrow \]
\[ \delta y = ln \left( \dfrac{x}{x - \delta x} \right) = ln\left( 1 +  \dfrac{x}{x + \delta x} \right) \]
\[ \downarrow \]
\[ \delta y \approx ln(1 + \varepsilon (x)) \approx \varepsilon(x) \]
\[ \downarrow \]
\[ \sigma_{ln} = \varepsilon\left( \dfrac{N - N_\text{фон}}{N_0 - N_\text{фон}} \right) = \varepsilon(N - N_\text{фон}) + \varepsilon(N_0 - N_\text{фон}) \]
\[ \downarrow \]

\begin{equation}
\sigma_{ln} = \dfrac{\sigma_N + \sigma_\text{Nфон}}{N - N_\text{фон}}    + \dfrac{\sigma_\text{No} + \sigma_\text{Nфон}}{N_0 - N_\text{фон}}
\label{lnmistake}
\end{equation}
\item По полученным данным построим кривую зависимости логарифма числа сосчитанных частиц от толщины для алюминия.

\begin{table}[H]
\caption{Данные для построения граффика. Алюминий}
\resizebox{\textwidth}{!}{
\begin{tabular}{|l|c|c|c|c|c|c|c|c|c|c|c|}
\hline
ln$\left(\dfrac{N_0-N_\text{фон}}{N-N_\text{фон}}\right)$& 0,41 & 0,94 & 1,42 & 1,88 & 2,36 & 2,83 & 3,26 & 3,71 & 4,19 & 4,66 & 5,18 \\ \hline
l, мм     & 20    & 40   & 60   & 80   & 100  & 120  & 140  & 160  & 180  & 200  & 220  \\ \hline
$\sigma_{ln}$             & 0,03 & 0,03 & 0,04 & 0,04 & 0,04 & 0,05 & 0,05 & 0,06 & 0,07 & 0,09 & 0,11 \\ \hline
\end{tabular}
}
\end{table}

\begin{figure}[H] 
\centering
\caption{График зависимости ln$\left(\dfrac{N_0-N_\text{фон}}{N-N_\text{фон}}\right)$ от l образца для Al}
    \includegraphics[width=.9\linewidth]{05}
\label{Al}
\end{figure}

\item Полученная зависимость аппроксимируется прямой вида $y = 0,0235 + 0,011$. 
\item Наши точки аппроксимируются  данной зависимостиь с достоверностью выше 90\% по методу $\phi^2$

\item Найдем ошибку вычисления коэффициента наклона (k) кривой по методу наименьших квадратов:
 
\[ k = (2,35 \pm 0,05)*10^{-2} \] 
\[ \varepsilon \approx 2\% \]

\item По полученным данным коэффициента наклона, найдем линейный коэффициента ослабления:

\[ \mu_{Al} = (0,235 \pm 0,005) \text{см}^{-1} \]
\[ \varepsilon(\mu) \approx 2\%\]

\item Используя линейный коэффициент ослабления железа выразим коэффициент $\mu'$. Из формул (1) и (2) следует:
	 
\[\mu l = \mu' m_1 \text{,}\]

где $m_1$ = $\rho \cdot l$. Отсюда:

	\begin{equation}
	\label{mu}
	\mu' = \dfrac{\mu_1}{\rho}\text{,}
	\end{equation}
где $\rho = 2,6989 \frac{\text{г}}{\text{см}^3}$	
	\[ \mu'_{Al} = 0,087 \frac{\text{см}^2}{\text{г}}  \]
	
\item Относительная прогрешность в таком случае будет сохраняться, а значит:
	\[ \mu'_{Al} = (8,7 \pm 0,2)*10^{-2} \frac{\text{см}^2}{\text{г}}  \]



\end{enumerate}

\subsection*{Исследование поглощений $\gamma-$ лучей в Свинце}

\begin{enumerate}

\item Проведем измерение числа частиц, попадающих в счетчик за время t = 10 секунд, при различных частицах поглотителя. Полученные данные выпишем в Таблицу 6 (k - колличество поглотителей, l - сумарная толщина в мм):


\begin{table}[H]
\caption{Поглотитель -- свинец, t = 10 секунд.} 
\centering
\begin{tabular}{|l|c|c|c|c|c|c|c|c|c|}
\hline
k штук   & 1     & 2     & 3    & 4    & 5    & 6    & 7   & 8   & 9   \\ \hline
l мм     & 5     & 10    & 15   & 20   & 25   & 30   & 35  & 40  & 45  \\ \hline
N частиц & 26872 & 13490 & 6783 & 3482 & 1784 & 1074 & 644 & 393 & 231 \\ \hline
N - Nфон & 26766 & 13384 & 6677 & 3376 & 1678 & 968  & 538 & 287 & 125 \\ \hline
\end{tabular}
\end{table}

\item Высчитывая кресты погрешости по формуле ($\ref{lnmistake}$), получим данные для построения графика зависимости:


\begin{table}[H]
\caption{Данные для построения граффика. Свинец.}
\centering
\begin{tabular}{|l|c|c|c|c|c|c|c|c|c|}
\hline
ln$\left(\dfrac{N_0-N_\text{фон}}{N-N_\text{фон}}\right)$ & 0,54 & 1,24 & 1,93 & 2,61 & 3,31 & 3,86 & 4,45 & 5,08 & 5,91 \\ \hline
l, мм                  & 5    & 10   & 15   & 20   & 25   & 30   & 35   & 40   & 45   \\ \hline
$\sigma_{ln}$            & 0,03 & 0,04 & 0,04 & 0,05 & 0,05 & 0,07 & 0,08 & 0,11 & 0,16 \\ \hline
\end{tabular}
\end{table}

\item По данным таблицы 7 построим граффик прямой аппроксимирующий, полученые значения: 

\begin{figure}[H] 
\centering
\caption{График зависимости ln$\left(\dfrac{N_0-N_\text{фон}}{N-N_\text{фон}}\right)$ от l образца для Pb}
    \includegraphics[width=.9\linewidth]{06}
\label{Al}
\end{figure}

\item Полученная зависимость аппроксимируется прямой вида $y = 0,131 + 0,059$. 
\item Наши точки аппроксимируются  данной зависимостиь с достоверностью выше 90\% по методу $\phi^2$

\item Найдем ошибку вычисления коэффициента наклона (k) кривой по методу наименьших квадратов:

\[ k = (1,31 \pm 0,03)*10^{-1}\] 
\[ \varepsilon \approx 2,3\% \]

\item По полученным данным коэффициента наклона, найдем линейный коэффициента ослабления в свинце:

\[ \mu_{Pb} = (1,31 \pm 0,03) \text{см}^{-1} \]
\[ \varepsilon(Pb) \approx 2\%\]

\item Используя линейный коэффициент ослабления железа и тот факт, что $ \rho = 11,3415 \dfrac{\text{г}}{\text{см}^3}$ выразим коэффициент $\mu'$ по формуле ($\ref{mu}$)
	
	\[ \mu'_{Pb} \approx 0,116 ~\frac{\text{см}^2}{\text{г}} \]
	
\item Относительная прогрешность в таком случае будет сохраняться, а значит:
	\[ \mu'_{Pb} = (1,16 \pm 0,026)*10^(-1) \frac{\text{см}^2}{\text{г}}  \]




\end{enumerate}

\subsection*{Исследование поглощений $\gamma-$ лучей в Железе}

\begin{enumerate}

\item Проведем измерение числа частиц, попадающих в счетчик за время t = 10 секунд, при различных частицах поглотителя. Полученные данные выпишем в Таблицу 10 (k - колличество поглотителей, l - сумарная толщина в мм):


\begin{table}[H]
\caption{Поглотитель -- железо, t = 10 секунд.} 
\centering
\begin{tabular}{|l|c|c|c|c|c|c|c|c|c|c|}
\hline
k штук   & 1     & 2     & 3    & 4    & 5    & 6    & 7   & 8   & 9   & 10  \\ \hline
l, мм     & 10    & 20    & 30   & 40   & 50   & 60   & 70  & 80  & 90  & 100 \\ \hline
N, частиц & 26005 & 12953 & 6611 & 3388 & 1853 & 1004 & 611 & 390 & 255 & 205 \\ \hline
N - Nфон & 25899 & 12847 & 6505 & 3282 & 1747 & 898  & 505 & 284 & 149 & 99  \\ \hline
\end{tabular}
\end{table}

\item Высчитывая кресты погрешости по формуле ($\ref{lnmistake}$), получим данные для построения графика зависимости:

\begin{table}[H]
\caption{Данные для построения граффика. Железо.}
\centering
\begin{tabular}{|l|c|c|c|c|c|c|c|c|c|c|}
\hline
ln$\left(\dfrac{N_0-N_\text{фон}}{N-N_\text{фон}}\right)$ & 0,58 & 1,28 & 1,96 & 2,64 & 3,27 & 3,94 & 4,51 & 5,09 & 5,73 & 6,14 \\ \hline
l, мм                    & 10   & 20   & 30   & 40   & 50   & 60   & 70   & 80   & 90   & 100  \\ \hline
$\sigma_{ln}$                & 0,03 & 0,04 & 0,04 & 0,05 & 0,05 & 0,07 & 0,08 & 0,11 & 0,15 & 0,19 \\ \hline
\end{tabular}
\end{table}

\item По данным таблицы 11 построим граффик прямой аппроксимирующий, полученые значения: 

\begin{figure}[H] 
\centering
\caption{График зависимости ln$\left(\dfrac{N_0-N_\text{фон}}{N-N_\text{фон}}\right)$ от l образца для Fe}
    \includegraphics[width=.9\linewidth]{07}
\label{Al}
\end{figure}

\item Полученная зависимость аппроксимируется прямой вида $y = 0,131 + 0,059$. 
\item Наши точки аппроксимируются  данной зависимостиь с достоверностью выше 90\% по методу $\phi^2$

\item Найдем ошибку вычисления коэффициента наклона (k) кривой по методу наименьших квадратов:

\[ k = (6,26 \pm 0,23)*10^2\] 
\[ \varepsilon \approx 3,7\% \]

\item По полученным данным коэффициента наклона, найдем линейный коэффициента ослабления железа:

\[ \mu_{Fe} = (0,626 \pm 0,023) \text{см}^{-1} \]

	\item Используя линейный коэффициент ослабления и знание о том, что $\rho = 	7,874 \dfrac{\text{г}}{\text{см}^3}$ железа выразим коэффициент $\mu'$ по формуле ($\ref{mu}$)
	
	\[ \mu'_{Fe} \approx 0,0795  ~\frac{\text{см}^2}{\text{г}} \]
	
\item Относительная прогрешность в таком случае будет сохраняться, а значит:
	\[ \mu'_{Fe} = (7,95 \pm 0,03)*10^{-2} \frac{\text{см}^2}{\text{г}}  \]


\end{enumerate}

\subsection*{Измерение средней энергии $\gamma$ - лучей}

\begin{enumerate}

\item Выпишем полученные значения $\mu$ и $\mu'$

\begin{table}[H]
\centering
\begin{tabular}{|c|c|c|c|}
\hline
                               & Al              & Pb              & Fe                \\ \hline
$\mu$, $cm^{-1}$                 & 0,235 $\pm$ 0,005 & 1,31 $\pm$ 0,03   & 0,626 $\pm$ 0,023   \\ \hline
$\mu'$, $\dfrac{cm^2}{\text{г}}$ & 0,087 $\pm$ 0,002 & 0,116 $\pm$ 0,003 & 0,0795 $\pm$ 0,0003 \\ \hline
\end{tabular}
\end{table}

\item Cопоставляя полученные данные с теми, что приведены в наших лабораторных практикумах: 

\begin{figure}[H] 
\centering
    \includegraphics[width=.9\linewidth]{gamma}
\label{Al}
\end{figure}

Получаем, что энергия $\gamma$ - квантов $E_{\gamma} \in [0,5; 0,6]$ МэВ

\item Полученный результат также можно подтвердить с помощью Рисунка 2 теоритического аспекта.

\subsection*{Дополнительное задание}

\item Проведем серию эспериментов по измерению прошедших сквозь поглотители частиц в зависимости от их расположени.

\item Соберем необходимые данные и запишем их в таблицы:


\begin{table}[H]
\centering
\begin{tabular}{|l|c|c|c|c|c|}
\hline
Свинец у щели            & 25454 & 25304 & 24996 & 24712 & 24910 \\ \hline
Свинец ближе к источнику & 26474 & 26400 & 26118 & 26291 & 25718 \\ \hline
\end{tabular}
\end{table}

\item Таким образом значение частиц попавших на счетчик (преодолев свинец), расчитанное по формулам ($\ref{fon}$) и ($\ref{pogfon}$) будет составлять:

у щели:	25075 $\pm$ 110; \\
у источника: 26200 $\pm$ 110;\\ 

\begin{table}[H]
\centering
\begin{tabular}{|l|c|c|c|c|c|}
\hline
Железо у щели            & 25188 & 24785 & 24954 & 25251 & 24815 \\ \hline
Железо ближе к источнику & 25121 & 25658 & 25264 & 25445 & 25212 \\ \hline
\end{tabular}
\end{table}

\item Таким образом значение частиц попавших на счетчик (преодолев железо), расчитанное по формулам ($\ref{fon}$) и ($\ref{pogfon}$) будет составлять:

у щели:	24999 $\pm$ 78; \\		
у источника: 25340	 $\pm$ 78;\\ 

\begin{table}[]
\centering
\begin{tabular}{|l|c|c|c|c|c|}
\hline
Алюминий у щели            & 30771 & 30354 & 30230 & 30384 & 30484 \\ \hline
Алюминий ближе к источнику & 31044 & 30443 & 30506 & 30554 & 30363 \\ \hline
\end{tabular}
\end{table}

\item Таким образом значение частиц попавших на счетчик (преодолев аюминий), расчитанное по формулам ($\ref{fon}$) и ($\ref{pogfon}$) будет составлять:

у щели:	30445	 $\pm$ 75; \\		
у источника: 30582	 $\pm$ 97;\\ 


\end{enumerate}

\section*{Вывод} 

\hspace{7mm} Иследование поглащения $\gamma$-лучей в свинце, алюминии и железе позволили нам подвердить теорию изложенную в данном лабораторном практикуме, так как достоверность апроксимации зависимости логарифма отноношения прошдших частиц ко всем от толщины частицам прямой линией состовляла свыше 90\% во всех трех случаях. К сожаоению, погрешности измеряемых величин в некоторых измерениях превышает 0,3\%. Это связано с нашим упущением в определении времени сбора даннных. Так же были получены такие значения как $\mu$ и $\mu'$, с помощью которых, нам удалось определить  среднюю энергию $\gamma$-лучей, испускаемых источником. Полученное значение составило $E_{\gamma} \sim 0,5\div0,6 ~\text{МэВ}$.

\hspace{7mm} Более того в рамках дополнительного эсперимента была выявлена следующая закономерность: Количество частиц, регистрируемых датчиком в случае, когда поглотитель стоит ближе к источнику, превышает данное значение при рассположении поглотителя ближе и датчику. Это может быть связано со следующем явлением: Частицы проходя некоторое расстояние до поглотителя возможно рассеиваются на этом пути на атомах воздуха, таким образом чем дольше путь до поглотителя, тем выше вероятность рассеивания на этом пути. Так и теряется некоторая незначительная доля частиц, которая вообще говоря не превышает погрешности измерения всех частиц выходящих из источника. 

\end{document}
