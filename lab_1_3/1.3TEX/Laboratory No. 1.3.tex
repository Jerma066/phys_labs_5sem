\documentclass{physlab}

\begin{document}
\include{cover}

\paragraph{Цель работы:} Исследовать энергетическую зависимость вероятности рассеяния электронов атомами ксенона, определить энергии электронов при которых наблюдается <<просветление>> ксенона и оценить размер его внешней электронной оболочки.

\section*{Теоритический аспект}
К. Рамзауэр в 1921 г. исследовал зависимость поперечных сечений упругого рассеяния электронов (с энергией до $10 \; \eV$) на атомах аргона. В результате этих исследований было обнаружено явление, получившее название эффекта Рамзауэра. \\

эффективное сечение реакции (иногда его называют поперечным сечением или просто сечением реакции) — это величина, характеризующая вероятность перехода системы двух сталкивающихся частиц в результате их рассеяния (упругого или неупругого) в определенное конечное состояние. Сечение $\sigma$ равно отношению числа $N$ таких переходов в единицу времени к плотности $nv$ потока рассеиваемых частиц, падающих на мишень, т. е. к числу частиц, проходящих в единицу времени через единичную площадку, перпендикулярную к их скорости $v$ ($n$ — плотность числа падающих частиц)

\begin{equation}
\sigma = \frac{N}{nv}
\end{equation}
Таким образом, сечение имеет размерность площади. \\

\begin{wrapfigure}{R}{0.4\linewidth}
\centering
    \includegraphics[width=.9\linewidth]{graph.png}
\caption{Качественная картина результатов измерения упругого рассеяния электронов в аргоне}
\label{graph}
\end{wrapfigure}

Качественно результат экспериментов Рамзауэра при энергии электронов порядка десятков электрон-вольт на аргоне показан на рис. \ref{graph}. По мере уменьшения энергии электрона от нескольких десятков электрон-вольт поперечное сечение его упругого рассеяния растет, как это и следует из очень простых рассуждений: чем меньше скорость электрона, тем медленнее он "проскакивает" мимо атома, тем больше время взаимодействия электронов с атомом и, тем самым, больше вероятность этого взаимодействия, т. е. сечение реакции. Однако в эксперименте наблюдалось, что при энергиях меньше 16 эВ сечение начинает уменьшаться, а при $E \sim $ 1 эВ практически равно нулю, т. е. аргон становится прозрачным для электронов. При дальнейшем уменьшении энергии электронов сечение рассеяния опять начинает возрастать. Объяснение этого эффекта требует  учета волновой природы электронов. \\


\begin{wrapfigure}{L}{0.4\linewidth}
\centering
    \includegraphics[width=.9\linewidth]{scheme.png}
\caption{Схема установки для измерения сечения рассеяния электронов в газах}
\label{scheme}
\end{wrapfigure}

Схема эксперимента Рамзауэра показана на рис. \ref{scheme}.

Пучок электронов, вылетая из накаленного катода К, проходит ускоряющую разность потенциалов V, приложенную между катодом и электродом э, и приобретает тем самым энергию $E = m v^2 /2 = eV$. При прохождении через газ часть электронов рассеивается на атомах, уходит в сторону и собирается коллектором КЛ, а прошедшие без рассеяния электроны попадают на анод А и создают анодный ток $I$. Ток $I$ пропорционален числу прошедших электронов, и поэтому непосредственно характеризует проницаемость газа для электронного пучка в зависимости от его скорости (ускоряющего напряжения). Согласно классическим воззрениям, с ростом напряжения V, как указывалось выше, сечение рассеяния уменьшается, и ток должен монотонно возрастать. \\

С точки зрения квантовой теории картина рассеяния выглядит иначе. Внутри атома потенциальная энергия налетающего электрона $U$ отлична от нуля, скорость электрона изменяется, становясь равной $v'$ в соответствии с законом сохранения энергии
\begin{equation}
E = \frac{m v^2}{2} = \frac{m v'^2}{2} + U,
\end{equation}
а значит, изменяется и его длина волны де Бройля. Таким образом, по отношению к электронной волне атом ведет себя как преломляющая среда с относительным показателем преломления
\begin{equation}
n = \frac{\lambda}{\lambda'} = \sqrt{1 - \frac{U}{E}}.
\end{equation}

\begin{wrapfigure}{R}{0.4\linewidth}
\centering
    \includegraphics[width=.9\linewidth]{problem.png}
\caption{Схематическое изображение прямоугольной ямы, над которой пролетает частица с энергией $E$}
\label{problem} 
\end{wrapfigure}

для качественного анализа вопроса рассмотрим следующую модель: будем считать, что электрон рассеивается на одномерной потенциальной яме конечной глубины. Форму ямы для качественных оценок можно считать прямоугольной. Модель прямоугольной потенциальной ямы является хорошим приближением для атомов тяжелых инертных газов, отличающихся наиболее компактной структурой и резкой внешней границей.

Уравнение Шредингера в данном случае имеет вид:
\begin{equation}
\psi'' + k^2 \psi = 0, 
\end{equation}
где
\begin{equation}
k^2 = 
\begin{cases}
k_1^2 = \frac{2mE}{\hbar^2} & \text{ - в областях I и III}\\
k_2^2 = \frac{2m(E+U_0)}{\hbar^2} & \text{ - в области II}
\end{cases}
\end{equation}

Коэффициент прохождения равен отношению квадратов амплитуд прошедшей и падающей волн и определяется выражением
\begin{equation}
D^{-1} = 1 + \frac{(k_1^2 - k_2^2)^2}{4k_1^2 k_2^2} \sin^2(k_2l) = 1 + \frac{U_0^2}{4E(E+U_0)} \sin^2(k_2l)
\end{equation}

Мы видим, что коэффициент прохождения частицы над ямой имеет, в зависимости от её энергии, ряд чередующихся максимумов и минимумов. В частности, если $k_2l = \pi$, то $\sin k_2l = 0$ и коэффициент прохождения равен единице, т. е. отраженная волна отсутствует, и электрон беспрепятственно проходит через атом, что является квантовым аналогом просветления оптики.
Таким образом, коэффициент прохождения электронов максимален при условии 
\[ k_2l = \sqrt{\frac{2m(E+U_0)}{\hbar^2}}l = \pi n, n = 1, 2, 3... \]

\begin{wrapfigure}{l}{0.4\linewidth} 
\centering
    \includegraphics[width=.9\linewidth]{wave.png}
\caption{Схема интерференции волн де Бройля при рассеянии на атоме}
\label{wave}
\end{wrapfigure}

это условие можно легко получить, рассматривая интерференцию электронных волн де Бройля в атоме. движущемуся электрону соответствует волна де Бройля, длина которой определяется соотношением $\lambda  = h/mv$. Если кинетическая энергия электрона невелика, то $E = mv^2/2$ и $\lambda = h/\sqrt{2mE}$. При движении электрона через атом длина волны де Бройля становится меньше и равна $\lambda' = h/\sqrt{2m(E + U_0)}$, где $U_0$ — глубина атомного потенциала. При этом, как показано на рис. \ref{wave}, волна де Бройля отражается от границ атомного потенциала, т.е. от поверхности атома, и происходит интерференция прошедшей через атом волны и волны 2, отраженной от передней и задней границы атома (эти волны когерентны).

Прошедшая волна 1 усилится волной 2, если геометрическая разность хода между ними $\Delta = 2l = \lambda'$, что соответствует условию первого интерференционного максимума, т. е. при условии
\begin{equation} \label{max}
2l = \frac{h}{\sqrt{2m(E_1 + U_0)}}
\end{equation}
Здесь $E_1$ -- энергия электрона, соответствующая этому условию.

С другой стороны, прошедшая волна ослабится, если $\Delta = 2l = (3/2)\lambda'$, т.е. при условии 
\begin{equation} \label{min}
2l = \frac{3}{2} \frac{h}{\sqrt{2m(E_2 + U_0)}}
\end{equation}

Решая совместно эти два уравнения, можно исключить $U_0$ и найти эффективный размер атома $l$
\begin{equation} \label{size}
l = \frac{h \sqrt{5}}{\sqrt{32m(E_2 - E_1)}}
\end{equation}

Понятно, что энергии $E_1$ и $E_2$ соответствует энергиям электронов, прошедших разность потенциалов $V_1$ и $V_2$, т.е. $E_1 = eV_1$ и $E_2 = eV_2$. 

Из формул (\ref{max}) и (\ref{min}) можно также по измеренным величинам $E_1$ и $E_2$ рассчитать эффективную глубину потенциальной ямы атома:
\begin{equation} \label{hole}
U_0 = \frac{4}{5} E_2 - \frac{9}{5} E_1
\end{equation}

\section*{Экспериментальная установка}

\begin{figure}[H] 
 \center{\includegraphics[width=0.7\linewidth]{tiratron.png}}
\caption{Схематическое изображение тиратрона (слева) и его конструкция (справа): 1, 2, 3~--~сетки; 4~--~внешний металлический цилиндр; 5~--~катод; 6~--~анод; 7~--~накаливаемая спираль}
\label{tiratron}
\end{figure}

Уравнение ВАХ тиратрона:
\begin{equation} \label{VAH}
I_a = I_0 e^{-C\omega(V)}, C = L n_a \Delta_a,
\end{equation}
где $I_0 = e N_0$ -- ток катода, $I_a = e N_a$~--~анодный ток, $\omega(V)$~--~вероятность рассеяния электрона на атоме, $\Delta_a$~--~площадь поперечного сечения атома, $n_a$~--~концентрация атомов газа в лампе, $N_0$~--~поток электронов у катода, $N_a$~--~поток электронов у анода, $L$~--~длина лампы.


Согласно классическим представлениям сечение рассеяния электрона на атоме должно падать монотонно с ростом V (обратно пропорционально скорости электрона, т. е. обратно пропорционально квадратному корню из его энергии), а значит, ВАХ будет монотонно возрастающей функцией, как это показано на рис. (\ref{omega})а. По квантовым соображениям вероятность рассеяния электронов и соответствующая ВАХ должны иметь вид, показанный на рис. (\ref{omega})б .

\begin{figure}[H] 
\centering
    \includegraphics[height=30ex]{omega.png}
\caption{Качественный вид вероятности рассеяния электрона атомом инертного газа и ВАХ тиратрона при классическом (а) и квантовом (б) рассмотрении}
\label{omega}
\end{figure}


Согласно формуле (\ref{VAH}) по измеренной ВАХ тиратрона можно определить зависимость вероятности рассеяния электрона от его энергии из соотношения:
\begin{equation}
\omega(V) = -\frac{1}{C} \log\frac{I_a(V)}{I_0}
\end{equation}

\section*{Ход работы}
\subsection*{Получение вольт-амперной характеристики тиратрона $I_k$ = f($V_a$)на экране осциллографа C1-83.}



\begin{enumerate}
\item Мы Проследили за ходом ВАХ тиратрона на экране ЭО при увеличении ускоряющего
напряжения $V_\text{катод-сетка}$ от 0 до max.
\item Перемещая сигнал ручками и меняя чувствительность канала Y, мы
добились размещения картин в центре экрана. 
\item Устанавливая напряжение накала лампы (ручка $V_\text{накала}$) в диапазоне 2,5 - 3,5 \V, Мы получили следующие изображения на экране осцилографа при значениях указанных в таблице 1:

\begin{table}[H]
\centering
\caption{ }
\begin{tabular}{|c|c|c|c|}
\hline
№ фото & Vнакала, В & Канал Y, мВ/дел  & Канал X, В/дел \\ \hline
фото 1 & $\sim$2,5  & 20               & 5				\\ \hline
фото 2 & $\sim$3    & 50               & 5				\\ \hline
фото 3 & $\sim$3,5  & 100              & 5				\\ \hline
\end{tabular}
\end{table}

\newpage 

\begin{figure}[H]
\caption{зависимость осцилограммы от напряжения накала.}
\begin{minipage}[h]{0.47\linewidth}
	\center{\includegraphics[width=1\linewidth]{1}} $\simeq$2,5 \V \\
\end{minipage}
\hfill
\begin{minipage}[h]{0.47\linewidth}
	\center{\includegraphics[width=1\linewidth]{2}} $\simeq$3 \V \\
\end{minipage}
\center{\includegraphics[width=0.50\linewidth]{3}} \\
	{$\simeq$3,5 \V}
\end{figure}

\item По полученным осцилограммам при максимальном ускоряющем напряжении рассчитаем размер электронной оболочки атома тремя способами, используя формулы (\ref{min}), (\ref{max}) и, искючив $U_0$ - глубинe атомного потенциала по формуле полученной из (\ref{size}). 

\begin{equation} \label{sz}
l = \frac{h \sqrt{5}}{\sqrt{32me(V_2 - V_1)}}
\end{equation}

Полуенные в ходе эксперимента значения:

\begin{table}[H]
\centering
\begin{tabular}{|c|c|c|c|c|}
\hline
№ фото & Vнач. отсчета, В & $V_{max}$, В & $V_{min}$, В & $\bigtriangleup$V, В \\ \hline
1      & 18,5                   & 5$\pm$1             & 11,5$\pm$1          & 6,5        \\ \hline
2      & 18,5                   & 6$\pm$1             & 11$\pm$1            & 5          \\ \hline
3      & 18,5                   & 5,5$\pm$1           & 10,5$\pm$1          & 5          \\ \hline
\end{tabular}
\end{table}

Для погрешности $\bigtriangleup$V из расмотрения трех значений посчитаем погрешность по формуле:
\begin{equation} \label{pog}
\sigma\bigtriangleup V = \sqrt{\sum_{i=1}^{n}(V_{ср}-V_i)^{2})/n}
\end{equation}


\[\sigma\bigtriangleup V \simeq 0,71\V \]

Тогда погрешность величины l для формул (\ref{max}) и (\ref{min}) будет равна:

\[\sigma l_{1} = 1/4 \frac{h (e\cdot\sigma V)}{\sqrt{2m} \sqrt{(E_1+U_0)}^{3}}  \]

\[\sigma l_{2} = 3/8 \frac{h (e\cdot\sigma V)}{\sqrt{2m} \sqrt{(E_2+U_0)}^{3}}  \]

A для формулы (\ref{sz}):

\begin{equation}\label{pogr}
\sigma l_{3} =  \frac{h \sqrt{5} (e\cdot \sigma\bigtriangleup V)}{\sqrt{32m} \sqrt{(E_2 - E_1)}^{3}}   
\end{equation}


Тогда полученные результаты приведены в таблице:

\begin{table}[H]
\centering
\begin{tabular}{|c|c|c|}
\hline
$l_1$, Анг. & $l_2$, Анг. & $l_3$, Анг.		 \\ \hline
2,5 $\pm$ 0,25    & 2,39$\pm$0,11     & 2,93 $\pm$ 0,41   \\ \hline
\end{tabular}
\end{table}

\hspace{7mm}Заметим, что полученные значения l размера электронной оболочки
атома инертного газа расччитанной в этом эксперименте совпадают в пределах погрешности. 

Расчет данной величины проводился при заданном значении величины $U_0$ - глубины атомного потенциала. Оценим его используя собранные нами значения по формуле (\ref{hole}):

\[ U_0 \simeq 1,5 \V \]


\item К сожалению нам не удалось оценить напряжение пробоя, так как наша установка не позволила нам наблюдать резкий скачок тока в конце кривой.

\newpage

\item Зафиксируем осицлограмму полученную на экране, поднеся к лампе постоянный магнит. Одно из полученных на осцилографе изображений:

\begin{figure}[H]
	\center{\includegraphics[width=0.50\linewidth]{4}} 
	\caption{Влияие магнита}
\end{figure}

\hspace{7mm}Стоит ометить, что влияние магнита координально зависит от оринтации магнита. Магнитное поле влияет на эфект Рамзауэра, так как оно отклоняет любой электрон, испытавший упругое столкновение. На фотографии видна ситуация в результате которой поток электронов в следвтиве действия магнитного поля не доходя до анода ушел на коллектор. 
Если поменять ориентацию, будет наблюдаться противоположный эффект - усиления размаха между максимумом и минимумом.

\hspace{7mm} Далее перейдем к обработке результатов полученных при измеренях в статическом режиме.

\end{enumerate}

\subsection*{Получение вольт-амперной характеристики $I_k$ = f($V_a$)в статическом режиме.}

\begin{enumerate}
\item Проведили измерения ВАХ тиратрона для 3-х значений напряжения накала.

Выпишем полученнныме данные в таблицы 2-4


\begin{table}[H]
\caption{Vнакала $\simeq$2,5 \V}
\resizebox{\textwidth}{!}{
\begin{tabular}{|c|c|c|c|c|c|c|c|c|c|c|c|c|c|c|c|c|c|}
\hline
№            & 1    & 2    & 3    & 4    & 5    & 6    & 7    & 8     & 9     & 10    & 11    & 12    & 13   & 14  & 15   & 16   & 17   \\ \hline
I анода, мкА & 0    & 0    & 0,75 & 13,6 & 16,6 & 17,2 & 17,6 & 17,78 & 17,45 & 15,76 & 14,86 & 13,51 & 10,8 & 8,8 & 8,2  & 8,5  & 9,3  \\ \hline
V катода, В  & 0,99 & 2,05 & 3,02 & 3,5  & 4,04 & 4,2  & 4,5  & 4,7   & 5,05  & 6,02  & 6,5   & 7,03  & 8,07 & 9,1 & 10,1 & 11,1 & 12,3 \\ \hline
\end{tabular}
}
\end{table}

\begin{table}[H]
\caption{Vнакала $\simeq$3 \V}
\resizebox{\textwidth}{!}{
\begin{tabular}{|c|c|c|c|c|c|c|c|c|c|c|c|c|c|c|}
\hline
№            & 1 & 2     & 3     & 4     & 5     & 6     & 7     & 8     & 9     & 10    & 11   & 12   & 13    & 14     \\ \hline
I анода, мкА & 0 & 0     & 0     & 22,5  & 53,9  & 63,67 & 69,3  & 73,78 & 72,79 & 67,08 & 59,6 & 55,5 & 55,45 & 59,3   \\ \hline
V катода, В  & 0 & 1,045 & 2,031 & 3,033 & 4,037 & 4,536 & 4,999 & 5,966 & 6,64  & 7,55  & 8,59 & 9,53 & 10,07 & 11,034 \\ \hline
\end{tabular}
}
\end{table}

\begin{table}[H]
\caption{Vнакала $\simeq$3,43 \V}
\resizebox{\textwidth}{!}{
\begin{tabular}{|c|c|c|c|c|c|c|c|c|c|c|c|c|c|c|c|}
\hline
№            & 1 & 2    & 3    & 4    & 5    & 6    & 7    & 8    & 9     & 10    & 11    & 12    & 13   & 14   & 15    \\ \hline
I анода, мкА & 0 & 0,01 & 33   & 61,3 & 71,1 & 76,6 & 79,7 & 81,3 & 81,24 & 80,73 & 78,02 & 71,25 & 65,6 & 64,7 & 69,02 \\ \hline
V катода, В  & 0 & 2,03 & 3,12 & 4,04 & 4,56 & 5,03 & 5,51 & 6,06 & 6,29  & 6,54  & 7,07  & 8,03  & 9,06 & 9,63 & 10,66 \\ \hline
\end{tabular}
}
\end{table}



\item По собранным данным построим графики зафисимости $I_a$ = f($V_c$) для статического режима.

\begin{figure}[H] 
\centering
    \includegraphics[height=60ex]{plot.png}
\caption{Зависимость $I_a$ = f($V_c$) }
\label{plot}
\end{figure}

Из графика получены следующие данные:

\begin{table}[H]
\centering
\begin{tabular}{|c|c|c|c|}
\hline
Vнакала & $V_{max}$, В & $V_{min}$, В & $\bigtriangleup$V, В \\ \hline
2,5           & 4,7           & 10,1          & 5,4        \\ \hline
3             & 5,96          & 10,07         & 4,11       \\ \hline
3,43          & 6,06          & 9,63          & 3,57       \\ \hline
\end{tabular}
\end{table}

\item Тогда, используя формулы (\ref{size}), (\ref{pog}) и (\ref{pogr}), получим следующие значение для l:

\[\sigma\bigtriangleup V = 0,64 \V\]
\[\bigtriangleup V = 4,76 \pm 0,64 \V\]
\[\sigma l = 0,4\text{Анг.}\]
\[l = 3 \pm 0,4\text{Анг.}\]


Cтоит отметить, что значение эффективного размера атома l полученного в статическом методе с учетом погрешности совпадает с числом полученным в динамическом.

\item Используя найднное нами значение для первого макимума в передыдущих опытах оценим положения следующих максимумов:
\begin{gather*}
k_2l = \sqrt{\frac{2m(E_n+U_0)}{\hbar^2}}l = \pi n, n = 1, 2, 3... \\
\Downarrow \\
\sqrt{\frac{E_n + U_0}{E_1 + U_0}} = n \\
\Downarrow \\
E_n = (E_1 + U_0) n^2 - U_0 \\
U_0 = 1.5 \eV \\
E_1 = 5.3 \eV \\ 
\Downarrow \\
E_2 = 25.7 \; \eV \\
E_3 = 59.7 \; \eV
\end{gather*}

Стоит отметить, что в диапозоне от 0 до 12 В лежить только одно значение энергии $E_1$, что мы и видим на них граффиках \ осцилограммах - только первый максимум. 

\item На основе Формулы (12) Найдем зависимость вероятности рассеяния электронов(с точностью до константы) от энергии и построим соответствующий график:

\begin{figure}[H] 
\centering
    \includegraphics[width=.9\linewidth]{ver}
\caption{Зависимость вероятности рассеяния электрона на атоме от его энергии}
\label{p(E)}
\end{figure}

\end{enumerate} 

\section{Вывод}

\hspace{7mm}В результате проведенных экспериментов было подтверждено, что описание рассеяния электрона на атоме является неточным, а именно имеет место эффект Рамзаура.
Нами были получены такие значения, как эффективный размер атома l, который во всех экмпериментах совпадал в пределах погрешности. К сожалению, нам не удалось найти табличного значения данной физической величины, однако, если сравнивать полученные нами значения с радиусом иона ксенона, которая равна 1,9 Ангстрем, то мы получаем величину равную по порядку.
Полученные значения:

\hspace{7mm} Также в ходе лабораторной работы нами была получена зависимость вероятности рассеяния электрона. Кототрая наглядно показывает зависимость вероятности взаимодествия электрона с атомом от энергии электрона.

\hspace{7mm} Более того, нами был получен и рассмотрен эффект влияния постоянного магнита а проведение эксперимента.

\end{document}