\documentclass{physlab}

\begin{document}
\include{cover}

\paragraph{Цель работы:} С помощью метода доплеровского сдвига мессбауэровской линии поглощения исследуем резонансное поглощение $\gamma$-лучей, испускаемых ядрами олова $^{119} \text{Sn}$ в соединении $\text{BaSnO}_3$ при комнатной температуре. Определим положение максимумами резонансного поглощения, его величину, а также экспериментальную ширину линии $\text{Г}_\text{экс}$. Оценим время жизни возбужденного состояния $^{119}\text{Sn}$.

\section*{Теоритический аспект}
Нуклоны (нейтроны и протоны) в атомном ядре, как и электроны в атоме, могут находиться в различных дискретных энергетических состояниях, или, как говорят, на различных энергетических уровнях. Самый низкий из уровней называется основным, остальные носят название возбужденных. Ядра, находящиеся в возбужденных состояниях, могут переходить на более низкие энергетические уровни, в том числе и на основной уровень. Такие переходы происходят самопроизвольно (спонтанно). Освобождающаяся энергия уносится фотоном. Так возникает $\gamma$-излучение. \\

\begin{wrapfigure}{R}{0.4\linewidth}
	\centering
	\includegraphics[width=.9\linewidth]{ris1}
	\caption{Энергетическое распределение, характеризующее возбужденное состояние ядра (а), и сдвиг линий испускания и поглощения из-за отдачи при свободных ядрах (б).}
	\label{ris1}
	
\end{wrapfigure}

В отличие от основного уровня, все возбужденные уровни ядра имеют конечную ширину. Отложим по оси абсцисс энергию ядра, а по оси ординат  вероятность найти ядро в состоянии с данной энергией (рис. \ref{ris1}а).
Ширина кривой, измеренная на половине
высоты, называется естественной шириной линии Г. Она связана со средним временем жизни $\tau$ возбужденного состояния ядра соотношением неопределенностей
\begin{equation}
\text{Г}\tau \simeq \hbar,
\label{eq1}
\end{equation}
где $\hbar$ - постоянная Планка. Неопределенность в энергии возбужденных уровней
приводит к появлению ширины у линий $\gamma$-излучения.

Ядра атомов могут не только испускать, но и поглощать фотоны. Если попадающий в атомное ядро фотон имеет энергию, равную разности энергий между основным и каким-либо возбужденным состояниями, то ядро может поглотить фотон и перейти в соответствующее возбужденное состояние. \\

Этот процесс возможен лишь для $\gamma$-лучей определенных энергий и носит, таким образом, резонансный характер. На первый взгляд резонансное поглощение $\gamma$-лучей должно представлять собой распространенное и легко наблюдаемое явление. Казалось бы, для его обнаружения достаточно пропустить поток $\gamma$-лучей, испущенных радиоактивным источником, через поглотитель, содержащий те же ядра в невозбужденном состоянии. На самом деле это не так. Дело в том, что энергия $E$, уносимая $\gamma$-квантом, оказывается меньше энергии $E_0$ перехода между уровнями. Небольшая, но вполне заметная доля энергии уносится ядром, которое вследствие отдачи начинает двигаться в сторону, противоположную направлению вылета
$\gamma$-кванта. Проведем некоторые простые оценки. Ядро, которое испускает
$\gamma$-квант, приобретает импульс отдачи, равный по абсолютной величине импульсу $\gamma$-кванта. Если ядро свободно и первоначально покоится, то
энергия отдачи $R$ равна
\begin{equation}
R = \dfrac{p^2}{M_\text{я}} = \dfrac{E_\gamma^2}{2M_\text{я}c^2}.
\label{eq2}
\end{equation}

Рассмотрим в качестве примера ядро олова $^{119}\text{Sn}$, у которого расстояние между основным и первым возбужденным уровнями равно $E_0 = 23,8~ \text{кэВ}$. Энергия отдачи в этом случае составляет
$$R = \dfrac{E_\gamma^2}{2M_\text{я}c^2}\simeq \dfrac{E_0^2}{2M_\text{я}c^2} = 2.5\cdot 10^{-3}~\text{эВ}.$$

Энергия, которая расходуется на отдачу ядра, поглощающего $\gamma$-квант, оказывается точно такой же. Эта картина иллюстрируется рис. \ref{ris1}б: линия испускания смещена на величину $R$ влево, а линия поглощения на столько же вправо от $E_0$.

Обсуждая влияние, которое оказывает сдвиг $R$ на резонансное поглощение $\gamma$-лучей, следует иметь в виду, что величина $R$ сама по себе не представляет существенного интереса. Важно соотношение между $R$ и шириной Г соответствующей резонансной линии. Резонансное поглощение возможно только в том случае, если спектры испускания и поглощения перекрываются, т. е. при условии
\begin{equation}
2R\leq\text{Г}
\end{equation}

Это условие почти никогда не выполняется для $\gamma$-переходов в свободных ядрах. Так, для рассмотренного ядра $^{119}\text{Sn}$ естественная ширина линии $\text{Г} \simeq 3 \cdot 10^{-8}~ \text{эВ}$, т.е. на много порядков величины меньше $R$. В принципе, можно компенсировать энергетический сдвиг $2R$ с помощью эффекта Доплера. Для этого излучающие и поглощающие ядра должны двигаться друг относительно друга со скоростью $V$ , равной
\begin{equation}
V = c\cdot 2R/E_\gamma
\label{eq4}
\end{equation}
Для ядер $^{119}\text{Sn}$ нужна скорость $V \simeq 60~\text{м/с}$.

В реальных условиях ширина линии испускания (и поглощения) складывается из собственной ширины линии и ее доплеровской ширины. Из этих двух ширин основную роль играет именно доплеровская ширина уровней, связанная с тепловым движением атомов. Произведем соответствующие оценки. Доплеровский сдвиг уровней $D$ можно
рассчитывать по нерелятивистским формулам, поскольку $v$  - тепловая скорость атомов много меньше скорости света:
\begin{equation}
D = \dfrac{v}{c}E_\gamma \simeq \dfrac{v}{c}E_0.
\label{eq5}
\end{equation}

Оценим величину $v$. Средняя кинетическая энергия, приходящаяся на одну степень свободы (движение по направлению к поглотителю или от него), равна $k_\text{Б}T/2$.  Имеем поэтому:
$$\dfrac{M_\text{я}v^2}{2} = \dfrac{k_\text{Б}T}{2},$$
или
\begin{equation}
v = \sqrt{k_\text{Б}T/\text{М}_\text{я}}.
\label{eq6}
\end{equation}
Подставляя это значение в формулу \eqref{eq5} и принимая во внимание \eqref{eq2}, найдем
$$D = \sqrt{2Rk_\text{Б}T}.$$
Более аккуратный расчет дает:
\begin{equation}
D = 2\sqrt{Rk_\text{Б}T}.
\label{eq7}
\end{equation}
При комнатных температурах $k_\text{Б}T \simeq 1/40~\text{эВ}.$ Для $^{119}\text{Sn}$ имеем:
$$D = 1.5\cdot 10^{-2}~\text{эВ}.$$

\begin{wrapfigure}{R}{0.4\linewidth}
	\centering
	\includegraphics[width=.9\linewidth]{ris2}
	\caption{Перекрытие линий испускания и поглощения вследствие доплеровского уширения.}
	\label{ris2}
\end{wrapfigure}
Мы видим, что доплеровская ширина линии существенно превосходит собственную ширину и в некоторых случаях (как, например, у $^{119}\text{Sn}$) оказывается больше сдвига $R$. В результате доплеровского уширения линии испускания и поглощения частично перекрываются, как это показано на рис. \ref{ris2}. Это означает, что существует некоторая доля $\gamma$-квантов, для которых отдача $R$ скомпенсирована и резонансное поглощение может, в принципе, наблюдаться, однако вероятность этого процесса очень мала.

Так происходит испускание и поглощение $\gamma$-квантов при ядерных переходах, если можно пренебречь энергией связи в веществе, т. е. считать ядра атомов свободными.

Перейдем теперь к исследованию процессов поглощения и испускания $\gamma$-квантов ядрами, входящими в состав кристаллической решетки. Наиболее прост - и наименее интересен - случай, когда энергия отдачи превышает энергию связи ядра в решетке. При этом связь ядра становится малосущественной и никаких новых явлений не наступает. Энергия, необходимая для смещения ядра, довольно велика: она
составляет $10 \div 30$ эВ. Формула \eqref{eq2} показывает, что рассматриваемый случай реализуется лишь при больших энергиях $\gamma$-квантов.

При испускании $\gamma$-квантов с $E < 1$ МэВ энергия отдачи оказывается недостаточной для вырывания ядра из кристаллической решетки, а импульс  в той или иной форме  передается всему кристаллу. Чаще всего энергия отдачи переходит в звуковые колебания решетки. Такой процесс перехода очевиден не только с квантовой, но и с классической точки зрения. В результате отдачи ядро сдвигается со своего места, но
не теряет связи с решеткой. Оно отдает свой импульс соседним ядрам, те передают его своим соседям  и в кристаллической решетке возбуждается звуковая волна. На квантово-механическом языке следует говорить, что энергия отдачи передается квантам звуковых колебаний - фононам.

Процесс генерации фононов (а значит, и процесс перехода импульса отдачи в звуковые волны) происходит тем легче, чем больше фононов уже имеется, т. е. при достаточно высоких температурах. При низких температурах этот процесс маловероятен. В этом случае все большую роль начинает играть бес фононный процесс, т. е. передача импульса отдачи всему кристаллу как целому. В формуле \eqref{eq2}, определяющей
энергию отдачи, вместо массы ядра следует теперь подставлять массу всего кристалла. Вследствие этого энергия отдачи понижается на $10 \div 20$ порядков и становится столь малой, что может считаться равной нулю.

Испускание и поглощение $\gamma$-квантов в твердых телах без рождения фононов носит название эффекта Мессбауэра. Его исследование является целью настоящей работы.

Теоретическое рассмотрение показывает, что вероятность эффекта Мессбауэра определяется выражением 
\begin{equation}
f = \exp(-4\pi^2\langle u^2\rangle/\lambda^2),
\label{eq8}
\end{equation}
где $\langle u^2 \rangle$ - среднеквадратичное смещение ядер в процессе тепловых колебаний решетки (в направлении вылета $\gamma$-кванта), $\lambda$ - длина волны
$\gamma$-излучения. Формула \eqref{eq8} подтверждает сделанный нами на основании качественных соображений вывод о том, что вероятность упругого испускания (и поглощения) $\gamma$-квантов уменьшается с температурой (с ростом $\langle u^2 \rangle$) и с ростом энергии перехода (с уменьшением длины волны $\lambda$).

Простой расчет, основанный на высказанных выше соображениях, показывает, что эффект Мессбауэра ограничен областью малых энергий $\gamma$-лучей ($\simeq 200$ кэВ). Оценить температурный интервал, в котором возможно наблюдение эффекта Мессбауэра, достаточно трудно. Как правило, наблюдение э екта Мессбауэра проводится при низких температурах, но в некоторых соединениях наблюдать его можно даже при комнатных температурах.

В заключение заметим, что линия резонансного (мессбауэровского) поглощения $\gamma$-лучей не размыта тепловым движением (которое представляет собой суперпозицию звуковых колебаний решетки) и имеет естественную, т. е. очень малую ширину. Эта резкая высокая линия четко выделяется на оне широких линий обычного резонансного поглощения $\gamma$-лучей.

Отсутствие доплеровского уширения линии испускания из-за беспорядочного теплового движения атомов в кристаллической решетке связано с тем, что частота тепловых колебаний составляет $10^{12} \div 10^{13} $, т. е. меняется с характерным временем порядка $10^{-12} \div 10^{-13}$ с, тогда как время жизни мессбауэровских ядерных уровней намного больше  $\simeq 10^{-7} \div 10^{-9}$ с. Поэтому за время испускания $\gamma$-кванта ядро успевает много раз изменить направление своей скорости и ее среднее значение оказывается практически равным нулю.

В данной работе эффект Мессбауэра изучается на ядрах олова $^{119}\text{Sn}$ в соединении $\text{BaSnO}3$. В этом соединении вероятность бесфононного излучения велика уже при комнатных температурах.

Гамма-излучение источника пропускается через резонансный поглотитель, т. е. поглотитель, содержащий стабильные ядра $^{119}\text{Sn}$. В резонансном поглотителе происходит взаимодействие $\gamma$-квантов как с атомными электронами за счет фотоэффекта и эффекта Комптона, так и с ядрами атомов. Поэтому интенсивность проходящего через поглотитель излучения уменьшается как
$\exp(-n_e\sigma_e)exp(-nf\sigma(E))\text{,}$
где $n_e$ и $n$ - число электронов и ядер поглотителя на $1~ \text{см}^2$ поглотителя соответственно, $f$  - вероятность эффекта Мессбауэра, а $\sigma_e$ и $\sigma(E)$ - соответственно сечение взаимодействия с электронами среды и сечение резонансного поглощения. Сечение резонансного поглощения имеет лоренцевскую орму кривой (формула Брейта-Вигнера):
\begin{equation}
\sigma(E)\propto \dfrac{(\text{Г}/2)^2}{(E - E_0)^2 + (\text{Г}/2)^2}.
\label{eq9}
\end{equation}

Здесь $E_0$ - энергия ядерного перехода, а Г - естественная ширина линии.

Излучение, прошедшее через поглотитель, регистрируется сцинтилляционным спектрометром, состоящим из счетчика и соединенной с ним электронной аппаратуры. Назначение этой аппаратуры заключается в том, чтобы выделить $\gamma$-лучи с нужной энергией, отбросив все постороннее излучение. Подробное описание сцинтилляционного спектрометра приведено в работе 5.3.

Наблюдение резонансного поглощения $\gamma$-лучей основано на методе доплеровского сдвига линий испускания и поглощения. Для создания этого сдвига поглотителю придается небольшая скорость. Мессбауэровская линия столь узка, что резонанс нарушается уже при ничтожных скоростях. Подставляя в формулу \eqref{eq4} вместо $R$ естественную ширину линии $\text{Г} = 3 \cdot 10^{-8}$ эВ, найдем, что для этого достаточна скорость порядка одного миллиметра в секунду! (Вместо 60 м/с, необходимых для получения резонанса на свободных ядрах.)

Если ядра источника и поглотителя находятся в идентичных кристаллах и при одинаковой температуре, то линия испускания полностью перекрывается с линией поглощения и максимальное поглощение наблюдается при нулевой скорости, как это изображено на рис.3.

\begin{figure}[H] 
	\centering
	\includegraphics[width=0.5\linewidth]{ris3}
	\label{ris3}
	\caption{Спектр упругого резонансного поглощения $\gamma$-квантов. Источник и поглотитель находятся в идентичных кристаллических решетках. Неупругое поглощение обусловлено главным образом взаимодействием $\gamma$-лучей с атомными электронами.}
\end{figure}
При испускании и поглощении $\gamma$-квантов ядрами, входящими в состав химических соединений, максимум линии поглощения может наблюдаться при скорости, отличной от нуля. Это объясняется тем, что энергия ядерного перехода, вообще говоря, зависит от электростатических сил взаимодействия ядра с окружающими его электронами. Вклад этого взаимодействия настолько мал, что непосредственно наблюдаться не может. Однако в опытах по изучению э екта Мессбауэра положение меняется. В этом случае влияние электростатических сил на энергетические уровни ядра должно сравниваться не с энергией $\gamma$-перехода
и даже не с энергией химической связи, а с шириной линии упругого резонансного поглощения. Смещение максимума линии при этом легко замечается. Это смещение называется химическим сдвигом. 

Для источника и поглотителя, находящихся в различных химических соединениях, максимум резонансного поглощения смещается относительно нулевой скорости на величину
\begin{equation}
v_p = \dfrac{\Delta E}{E_0}c.
\label{eq10}
\end{equation}
Обычно в опыте по резонансному поглощению определяется величина <<амплитуды>> эффекта 
\begin{equation}
\varepsilon(v) = \dfrac{N(\infty) - N(v)}{N(\infty) - N_\text{ф}},
\label{eq11}
\end{equation}
где $N(v)$ - скорость счета квантов, прошедших через поглотитель при некоторой скорости $v$, $N(\infty)$ - скорость счета квантов при достаточно большой скорости, когда резонансное поглощение отсутствует, $N$ - скорость счета радиоактивного фона. Величина $\varepsilon(v)$ является относительной величиной, она безразмерна и не зависит ни от активности источника, ни от величины поглощения излучения электронами.

Измеряемая на опыте ширина резонансной линии $\text{Г}_\text{эксп}$ - результат наложения линий источника и поглотителя. При тонких поглотителях и источниках и при отсутствии вибраций ширина линии равна удвоенной естественной ширине $2\text{Г}$ (см. рис.3). Увеличение толщины поглотителя заметно уширяет резонансную линию, что обусловлено несколькими причинами: 1) кванты, энергия которых лежит вблизи максимума линии, сильно поглощаются уже в тонком поглотителе, и для них увеличение толщины поглотителя сказывается слабее, чем на
<<крыльях>> линии; 2) уширение линии может происходить и вследствие самопоглощения квантов в источнике, если в нем содержатся резонансно поглощающие ядра; 3) существенный вклад в ширину линии дает так называемое аппаратурное уширение, связанное с несовершенством измерительной аппаратуры, в частности, с вибрациями источника и поглотителя (доплеровское уширение); 4) уширение линии может вызываться неравномерностью скорости перемещения поглотителя относительно источника.

\newpage

\section*{Экспериментальная установка}

Блок-схема экспериментальной установки приведена на рис.. Поглотителем служит оловянная фольга (или соединение, содержащее олово).
\begin{figure}[H] 
	\centering
	\includegraphics[width=0.8\linewidth]{ris4}
	\label{ris4}
	\caption{Блок-схема установки для наблюдения эффекта Мессбауэра: Э - эксцентрик, С - сцинтилляционный кристалл NaI(Tl), У - усилитель, АА - одноканальный амплитудный анализатор, ЭВМ - персональный компьютер, Г - генератор для питания двигателя, РД-09 - двигатель с редуктором, ВСВ - высоковольтный стабилизированный выпрямитель.}
\end{figure}

Поглотитель укреплен в рамке, которая приводится в движение кулачковым механизмом. Форма эксцентрика выбрана так, чтобы движение поглотителя происходило с постоянной скоростью (при равномерном вращении эксцентрика).

Экспериментально измеряемая зависимость перемещения рамки от угла поворота эксцентрика показана на рис. 5. Из графика видно, что движение с постоянной скоростью происходит в достаточно широком диапазоне углов. Измерения следует производить только на участках с равномерным движением. Выбор времени измерений производится автоматически с помощью двух тефлоновых кулачков, управляющих моментами включения и выключения установки. При замыкании концевых выключателей один из них подает сигнал начала движения с постоянной скоростью, а другой сигнализирует об окончании движения на линейном участке. Величина линейного участка $S$ указана на установке.

Скорость перемещения поглотителя можно варьировать в диапазоне от $0.1 до 5.0~ \text{мм/с}$, меняя угловую скорость вращения эксцентрика при помощи двигателя с редуктором Д-09 (редукция 1/137). Для уменьшения вибраций узел с эксцентриком и двигатель связаны ременной передачей и укреплены на капитальной стене независимо друг от друга. Двигатель питается от специального генератора.

\begin{wrapfigure}{R}{0.4\linewidth}
	\centering
	\includegraphics[width=.9\linewidth]{ris5}
	\caption{Зависимость перемещения поглотителя от угла поворота эксцентрика.}
	\label{ris5}
\end{wrapfigure}

В данной работе в качестве источника $\gamma$-квантов используется радиоактивный
изотоп олова $^{119m}\text{Sn}$ в виде соединения $\text{BaSnO}3$. Изомер $^{119m}\text{Sn}$ живет 250 дней и распадается с излучением $\gamma$-квантов с энергией 65 кэВ, переходя на короткоживущий первый возбужденный уровень (рис. 6а).

При каскадном переходе с первого уровня на основной излучаются используемые в работе $\gamma$-лучи с энергией 23,8 кэВ. Источник получен в ядерном реакторе при
облучении нейтронами образца олова, содержащего 96\% изотопа $^{118}\text{Sn}$. Переход с энергией 65 кэВ сильно конвертирован. Испускание электронов внутренней конверсии вызывает появление характеристического рентгеновского излучения с $E_X = 25,4$ кэВ. Это излучение образует <<фон>>, мешающий измерениям. Этот фон можно значительно подавить с помощью характеристического фильтра из палладия. Край K-полосы поглощения палладия приходится на энергию $E_K = 24,3$ кэВ. Поэтому палладиевый фильтр сильно поглощает рентгеновское излучение олова ($E_X > E_K$) и мало ослабляет поток исследуемых $\gamma$-квантов ($E_0 < E_K$). Палладиевая фольга толщиной 60 мкм приклеена непосредственно на источнике. Источник в металлическом контейнере укреплен неподвижно над поглотителем. Как правило, источники $\gamma$-излучения содержат радиоактивные примеси, которые дают заметный вклад в суммарное излучение.

\begin{figure}[H] 
	\centering
	\includegraphics[width=0.8\linewidth]{ris6}
	\label{ris6}
	\caption{Схема распада $^{119}\text{Sn}$ (a); спектр излучения источника $\text{BaSnO}_3$, снятый с помощью сцинтилляционного спектрометра (б).}
\end{figure}

На рис. 6б показан спектр излучения источника $\text{BaSnO}_3$, снятый с помощью калиброванного по энергии сцинтилляционного спектрометра с <<толстым>> (40 мм) кристаллом $NaI(Tl)$. Максимумы в спектре соответствуют фотоэлектрическому поглощению $\gamma$-квантов различных энергий в кристалле $NaI(Tl)$. Они называются фотопиками (или пиками полного поглощения). В спектре источника кроме основной (мессбауэровской) линии $\simeq 24$ кэВ присутствуют кванты с энергиями
$\simeq 100$ кэВ и $\simeq 300$ кэВ. Для наблюдения эффекта необходимо выделить основную линию из общего излучения. Этого достигают, устанавливая <<окно>> амплитудного анализатора спектрометра на фотопик линии 23,8 кэВ.

Сцинтилляционный спектрометр, используемый в работе, несколько отличается от спектрометра, описанного в работе 5.3. Вместо <<толстого>> применяется <<тонкий>> (1.0 мм) кристалл $NaI(Tl)$, который хорошо поглощает излучение с энергией $20 \div 30$ кэВ и малочувствителен к $\gamma$-квантам больших энергий. Фотоэлектронный умножитель специально подобран и имеет малые собственные шумы (меньше 2 кэВ).

Кристалл и ФЭУ находятся в отдельном светонепроницаемом выносном блоке (см. рис. 4). Выносной блок устанавливается на штативе под рамкой с поглотителем.

Данный спектрометр откалиброван заранее с помощью эталонных источников $\gamma$-излучения (аналогично тому, как это делается в работе 5.3). Если на ФЭУ подано напряжение, указанное на установке, то пик фотоэлектрического поглощения $\gamma$-квантов с энергией 23,8 кэВ должен быть расположен в интервале значений порогов амплитудно го анализатора от 1 до 10 В. Прошедшие через анализатор импульсы поступают для регистрации в ЭВМ.

Канал счета времени и канал счета импульсов на ЭВМ выведены на реле, управление которым ведется от тефлоновых кулачков. Под их действием концевые выключатели замыкают или размыкают токовое питание электромагнита реле. Таким образом, схема автоматически включается на участке, где поглотитель движется с постоянной скоростью, и выключается, когда этот участок пройден. Скорость считается <<положительной>>, если поглотитель движется навстречу источнику. При движении в обратном направлении скорость <<отрицательна>>. При работе установки абсолютное значение скорости поглотителя указывается на дисплее ЭВМ, поскольку ЭВМ фиксирует время, за которое поглотитель проходит известный линейный участок.

Смена скорости или ее подбор производится путем изменения частоты генератора, питающего двигатель; при этом следует также переключать согласующую емкость на двигателе. Диапазон изменения частоты  от 0 до 200 Гц, а емкости  от 0,1 до 0,4 мкФ. Чтобы вращать вал двигателя с большой скоростью, следует включать малую емкость и большую частоту. Для медленного вращения нужна малая частота и большая емкость. Измерять частоту и емкость при этом не нужно, так как скорость определяется путем независимого измерения, как это было указано выше.

Если опыт проводится при неподвижном поглотителе, то, вращая рукой шкив, нужно выбрать такое его положение, при котором включено реле, о чем свидетельствует светодиод, расположенный вблизи эксцентрика.

\newpage

\section*{Ход работы}

\subsection*{Анализ спектра источника}

\begin{enumerate}

\item Измерение спектра источника

\hspace{7mm}Настроим анализатор на пик фотопоглащения $\gamma$-квантов с энергией 23,8 кэВ. Полученные значения занесем в таблицу 1 и изобразим на рисунке 1 зависимость интенсивности N от нижнего порога V. Для наглядности будем рассматривать значения при V принадлежащих от 2 до 9,5.

\begin{table}[H]
\caption{Измерение спектра источника}
\resizebox{\textwidth}{!}{
\begin{tabular}{|c|c|c|c|c|c|c|c|c|c|c|c|c|c|c|c|c|c|c|c|c|}
\hline
№               &  1  & 2   & 3     & 4     & 5    & 6    & 7    & 8    & 9  & 10   & 11   & 12   & 13 & 14   & 15   & 16  & 17   & 18   & 19   & 20  \\ \hline
Нижний порог, V & 0 & 0,5 & 1     & 1,5   & 2    & 2,5  & 3    & 3,5  & 4  & 4,5  & 5    & 5,5  & 6  & 6,5  & 7    & 7,5 & 8    & 8,5  & 9    & 9,5 \\ \hline
Интесивность, N & 0 & 0   & 253,2 & 166,4 & 20,4 & 12,6 & 16,6 & 30,4 & 32 & 48,2 & 62,6 & 63,8 & 65 & 59,2 & 47,8 & 28  & 24,2 & 13,6 & 10,4 & 6,2 \\ \hline
\end{tabular}
}
\end{table}

\begin{figure}[H] 
\centering
\caption{Измерение спектра источника}
    \includegraphics[width=.9\linewidth]{graph1}
\label{W(T)}
\end{figure}

\hspace{7mm} Таким образом в ходе эксперимента мы получили график колоколообразной формы с фоновым сигналом в области низкх напряжений. Учитывая тот факт, что данные записываются при V $\in$ [2; 9,5], Колокол смешен в правую часть. Что подтверждает теорию описанную в лабораторном практикуме.

\subsection*{Измерение спектра резонансного поглощения}

\item Измерения для образцов проводились в последовательности изложенной в лаб. практикуме. (образцы: №1 $\Rightarrow$ №4 $\Rightarrow$ №2 $\Rightarrow$ №3)


\begin{table}[H]
\centering
\caption{Данные поглотителей}
\begin{tabular}{|l|c|c|c|c|}
\hline
Положение держателя & 1  & 2   & 3   & 4       \\ \hline
Поглотитель         & Sn & Sn  & Sn  & Sn$O_2$ \\ \hline
Толщина Поглотителя & 90 & 180 & 310 & --      \\ \hline
\end{tabular}
\end{table}

\item Собранные данные аппроксимируем Лоренцианом для дальнейшего рассмотрения c помощью программы "Origin":

\begin{table}[H]
\caption{Образец №1}
\resizebox{\textwidth}{!}{
\begin{tabular}{|c|c|c|c|c|c|c|c|c|c|c|c|c|}
\hline
№  & 1     & 2     & 3     & 4     & 5     & 6     & 7     & 8     & 9     & 10    & 11    & 12    \\ \hline
V- & 0     & 0,4   & 1     & 1,82  & 1,34  & 2,78  & 2,32  & 3,19  & 4,19  & 5,09  & 5,6   & 5,8   \\ \hline
I- & 128,9 & 136,2 & 135,5 & 136   & 133,9 & 146,5 & 141,1 & 136,4 & 131,7 & 137,2 & 137,4 & 138   \\ \hline
V+ & 0     & 0,44  & 1,09  & 1,97  & 1,44  & 2,94  & 2,48  & 3,38  & 4,46  & 5,42  & 5,98  & 6,18  \\ \hline
I+ & 128,9 & 144,2 & 136,8 & 133,2 & 137   & 123,4 & 117,4 & 134   & 138,1 & 141,4 & 139,2 & 140,2 \\ \hline
\end{tabular}
}
\end{table}

\begin{figure}[H] 
\centering
\caption{образец №1}
    \includegraphics[width=.9\linewidth]{lor1}
\label{W(T)}
\end{figure}

\begin{figure}[H] 
\centering
\caption{образец №1}
    \includegraphics[width=.5\linewidth]{lor11}
\label{W(T)}
\end{figure}


\begin{table}[H]
\caption{Образец №4}
\resizebox{\textwidth}{!}{
\begin{tabular}{|c|c|c|c|c|c|c|c|c|c|c|}
\hline
№  & 1      & 2     & 3     & 4     & 5     & 6     & 7     & 8     & 9     & 10    \\ \hline
V- & 0      & 0,79  & 1,51  & 2,13  & 2,35  & 3,25  & 3,95  & 4,61  & 5,11  & 0,29  \\ \hline
I- & 3474,4 & 222   & 239,4 & 250,7 & 259,8 & 259   & 258   & 260,5 & 271,3 & 195,3 \\ \hline
V+ & 0      & 0,88  & 1,64  & 2,29  & 2,5   & 3,46  & 4,2   & 4,89  & 5,43  & 0,34  \\ \hline
I+ & 3474,4 & 224,4 & 250,4 & 262   & 263   & 263,2 & 270,4 & 267,3 & 266,4 & 199,9 \\ \hline
\end{tabular}
}
\end{table}

\begin{figure}[H] 
\centering
\caption{образец №4}
    \includegraphics[width=.9\linewidth]{lor4}
\label{W(T)}
\end{figure}

\begin{figure}[H] 
\centering
\caption{образец №4}
    \includegraphics[width=.6\linewidth]{lor22}
\label{W(T)}
\end{figure}

\begin{table}[H]
\caption{Образец №2}
\resizebox{\textwidth}{!}{
\begin{tabular}{|c|c|c|c|c|c|c|c|c|c|c|c|c|}
\hline
№  & 1     & 2     & 3     & 4     & 5     & 6     & 7     & 8     & 9     & 10    & 11    & 12    \\ \hline
V- & 0     & 0,23  & 0,93  & 1,68  & 2     & 2,5   & 2,94  & 3,43  & 4,28  & 4,73  & 5,32  & 5,59  \\ \hline
I- & 284,5 & 282,8 & 290,3 & 293,3 & 292,6 & 292,9 & 294,2 & 297,7 & 294,5 & 292   & 291,3 & 292,3 \\ \hline
V+ & 0     & 0,32  & 1,01  & 1,79  & 2,14  & 2,68  & 3,11  & 3,66  & 4,54  & 5,04  & 5,66  & 5,98  \\ \hline
I+ & 284,5 & 302,1 & 295,5 & 292,4 & 274,7 & 264,0 & 275,8 & 296,6 & 299,5 & 292,6 & 299,6 & 304,5 \\ \hline
\end{tabular}
}
\end{table}

\begin{figure}[H] 
\centering
\caption{образец №2}
    \includegraphics[width=.9\linewidth]{lor2}
\label{W(T)}
\end{figure}

\begin{figure}[H] 
\centering
\caption{образец №2}
    \includegraphics[width=.6\linewidth]{lor44}
\label{W(T)}
\end{figure}

\begin{table}[H]
\caption{Образец №3}
\resizebox{\textwidth}{!}{
\begin{tabular}{|c|c|c|c|c|c|c|c|c|c|c|c|c|c|}
\hline
№  & 1  & 2    & 3    & 4    & 5    & 6    & 7    & 8    & 9    & 10   & 11   & 12   & 13   \\ \hline
V- & 0  & 0,3  & 0,75 & 1,27 & 1,83 & 2,34 & 3,23 & 2,79 & 3,92 & 4,55 & 5,13 & 5,48 & 5,71 \\ \hline
I- & 95 & 77,1 & 83,1 & 82,9 & 86,3 & 84   & 86,1 & 83,7 & 81,2 & 84,8 & 87,3 & 84   & 79,5 \\ \hline
V+ & 0  & 0,45 & 0,8  & 1,37 & 1,95 & 2,49 & 3,44 & 2,96 & 4,17 & 4,82 & 5,45 & 5,84 & 6,11 \\ \hline
I+ & 95 & 85,1 & 82,3 & 85,7 & 82,9 & 72,9 & 83,6 & 81,7 & 88,8 & 87,7 & 89,1 & 83,8 & 84,6 \\ \hline
\end{tabular}
}
\end{table}

\begin{figure}[H] 
\centering
\caption{образец №3}
    \includegraphics[width=.9\linewidth]{lor3}
\label{W(T)}
\end{figure}

\begin{figure}[H] 
\centering
\caption{образец №2}
    \includegraphics[width=.6\linewidth]{lor33}
\label{W(T)}
\end{figure}

\item Полученные таким образом данные позволяют нам определим с помощью формул \eqref{eq10} и \eqref{eq11} амплитуду резонансного поглощения в максимуме $\varepsilon(v)$, величину химического сдвига $\Delta E$ и экспериментальную ширину линии $\text{Г}_\text{экс}$. 

\item Занесем полученные данные в таблицу:

Так как в выведенных "Origin" таблицах H = 2A/($V \cdot \pi$ ), то значение $\varepsilon(v) = \dfrac{-H}{y_0}$; $D = \dfrac{x_c}{c} \cdot E_0$, где $E_0$ = 23,8 кэВ, а c - скорость света; $\text{Г}_\text{эксп} = \dfrac{W}{2}$ 

\begin{table}[H]
\centering
\caption{Конечные рзультаты опытов}
\begin{tabular}{|c|c|c|c|c|}
\hline
Образец            & 1               & 2               & 3               & 4                \\ \hline
$\varepsilon$ в \% & 16 $\pm$ 5      & 13 $\pm$ 5      & 15 $\pm$ 5      & 29 $\pm$ 5       \\ \hline
$\Delta$E,  мм/c           & 2,48 $\pm$ 0,08 & 2,58 $\pm$ 0,08 & 2,49 $\pm$ 0,07 & 0,05 $\pm$ 0,003 \\ \hline
$\Delta$E*$10^{-7}$ , эВ     & 1,97 $\pm 0,003$            & 2,05 $\pm 0,03$                      & 1,98  $\pm 0,004$                    & 0,04 $\pm 0,002$            \\ \hline
Г,  мм/c           & 0,26 $\pm 0,014$           & 0,44 $\pm 0,012$           & 0,6 $\pm 0,013$             & 1,01 $\pm 0,017$             \\ \hline
Г*$10^{-8}$,  эВ     & 2,06 $\pm 0,4$            & 3,49 $\pm 0,54$           & 4,76 $\pm 0,6$           & 8,01  $\pm 0,7$           \\ \hline
\end{tabular}
\end{table}


\end{enumerate}

\section*{Вывод}

\hspace{7mm} Таким образом собранные и обработанные нами данные сходятся по порядку велицины со значениями указанными в лабораторном пркатикуме. А именно значение измеренной ширины резонансной линии $Г_{\text{эксп.}}$ = $3*10^{-8}$ и значения полученные  нами в ходе аппроксимации в Таблице 7. Более того анализ спектра источника позволил нам убедиться в корекктности работы оборудования, что позволило нам проводить все вышеупомянтуые опыты.

\end{document}
