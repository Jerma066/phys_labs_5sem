\documentclass{physlab}

\begin{document}
\include{cover}

\paragraph{Цель работы:} Вычислить магнитные моменты протона, дейтрона и ядра фтора на основе измерения их $g-$факторов методом ядерного магнитного резонанса (ЯМР). Полученные данные сравнить с вычислениями магнитных моментов на основе кварковой модели адронов и одночастичной оболочечной модели  ядер.
	

\section{Теория}
Отношение $\gamma$ магнитного момента к механическому называется гиромагнитным отношением:
\begin{equation}
\vec{\mu} = \gamma \vec{M}.
\end{equation}

Зачастую, вместо $\gamma$ используют более простую величину - $g$-фактор. Он также является отношением магнитного момента к механическому, но при этом магнитный момент измеряется в ядерных магнетонах Бора ($\mu_\text{я} = e \hbar / 2 m_p c$), а механический момент -- в единицах $\hbar$:
\begin{equation}
g = \frac{\mu / \mu _\text{я}}{M/\hbar} = \frac{\mu}{\mu_\text{я}} \frac{\hbar}{M} = \frac{\hbar}{\mu_\text{я}} \gamma.
\end{equation}

Отсюда 
\begin{equation} \label{mu}
\vec{\mu} = \frac{\mu_\text{я}}{\hbar} g \vec{M}.
\end{equation}


Проецируя $M$ и $\mu$ на направление вектора $B$, получаем:
\begin{equation}
\vec{\mu_B} = \frac{\mu_\text{я}}{\hbar} g \vec{M_B} = \mu_\text{я} g m.
\end{equation}
Наибольшее значение $\mu_B$ равно $\mu_\text{я} g I$. Его принято называть магнитным моментом ядра. 

Расстояние между двумя соседними компонентами расщепившегося в магнитном поле уровня:
\begin{equation} \label{deltaE}
\Delta E = B \Delta \mu_B = B \mu_\text{я} g \Delta m = B \mu_\text{я} g.
\end{equation}

Между компонентами расщепившегося уровня могут происходить электромагнитные переходы. Энергия квантов при этом определяется выражением (\ref{deltaE}), и явление носит резонансный характер. Частота излучения:

\begin{equation} \label{nu}
\nu = \frac{\Delta E}{h} = \frac{B \mu_\text{я} g}{h}.
\end{equation}

Возбуждение переходов между компонентами расщепившегося ядерного уровня~---~ядерный магнитный резонанс (ЯМР). 

В данной работе $g$-фактор определяется с помощью явления ЯМР. Изменяя частоту переменного магнитного поля, мы можем найти положение максимума поглощения, т.е. частоту резонанса. По этому максимуму определяется $g$-фактор из соотношения (\ref{nu}).
	
\section{Экспериментальная установка}
	
\begin{figure}[H]
 \centering
 {\includegraphics[width=0.6\linewidth]{scheme.png}}
\caption{Схема установки: 1~---~часть индикаторной установки, 2~---~исследуемый образец, 3~---~трансформатор, 4~---~электромагнит, 5 - катушки, 6 - модулирующие катушки, 8 - потенциометр }
\end{figure}

\section{Ход работы}



\hspace{7mm}Убедившись в готовности оборудования к проведению эсперимента, начнем, 	
помещая разные образцы между полюсами электромагнита и устанавливая частоту $f_0$ индикаторной установки  $\sim 10$ МГц, плавно менять магнитное поле в зазоре электромагнита, пока не обнаруживали сигнал ЯМР для образцов в следующем порядке.

\begin{table}[H]
\centering
\caption{Порядок выбора образцов в проводимом эксперименте}
\begin{tabular}{|c|c|c|c|c|}
\hline
номер        & №3       & №1       & №2     & №5                \\ \hline
образец      & вода     & резина   & тефлон & тяжелая вода($D_2$O) \\ \hline
ЯМР на ядрах & водорода & водорода & фтора  & дейтерия          \\ \hline
\end{tabular}
\end{table}

\subsection*{Ядерный магнтный резонанс на ядрах водорода}

\begin{enumerate}

\item Результаты, полученые при опыте с \underline{образоцом №3 - водой}.

\item Ток в катушке электромагнита, установленный в соотвествии с калибровачным граффиком зависимости магнитного поля от тока в кашутках, соотвествующий диапозону значений $B\in[225;245] \text{мТл}$ --  J = 1,8 А.

\item Получен следующий сигнал Ядерного Магнитного Резонанса на на осцилографическом идтикаторе прибора Ш1-9:

\begin{figure}[H]
 \centering
 \caption{сигнал ЯМР на ядрах водобора - образец №3}
 {\includegraphics[width=0.6\linewidth]{photo1}}
\end{figure}

\item Показания частотометра -- значение резонансной частоты $f_0 = 9,88230 \text{ МГц}$

\item Показание датчика Хола на экране прибора Ш1-10 $B_0 = 230 \text{ мТл}$

\item Таким образом значения g-фактора ядер водорода в данном эскиперменте:

\begin{equation}
\label{gform}
g_\text{я} = \frac{\hbar \omega_0}{\mu_\text{я} B_0} = \frac{h f_0}{\mu_\text{я} B_0} \text{,}
\end{equation}

где $\mu_\text{я} = \dfrac{e\hbar}{2Mc}$, M - масса протона. $\mu_\text{я} \approx 0,50504 \cdot 10^{-23} \left[\dfrac{\text{эрг}}{\text{Гс}}\right] $.\\

\[g_\text{ядер водорода} = 5,694 \]

\item Рассчитаем ошибку данного измеения по формуле:

\begin{equation}
\label{sigma_g}
\sigma_g = \sqrt{\left(\sigma_f \cdot \frac{h}{\mu_\text{я} B_0}\right)^2 + \left(\sigma_B \cdot \frac{hf_0}{\mu_\text{я} B_0^2}\right)^2} \text{,}
\end{equation} 

где будем считать, что $\sigma_f \approx $ 10 Гц, и $\sigma_B \approx $ 10 мТл. Первая величина погрешности выбрана как последнее значащие число на индикаторах прибора, определяющего эту величину. В то время как второе, менялось в зависимости от расположение моей руки примерно на указанные 10 мТл.

\[\sigma_g \approx 0,025\]

\item Таким образом:

\[g = 5,694 \pm 0,25 \]
\[\varepsilon(g) =  4,4\%\]


\item Учитывая, что угловой момент протона определяется только его спином, рассчитаем момент протона по следующей формуле:

\begin{equation}
	\label{mu}
    \mu = g_\text{я} \mu_\text{я}I \text{,}
\end{equation}

где I -- полный момент количества движения ядра. Для водорода I = $\dfrac{1}{2}$

\item Таким образом найдем ведичину $\mu$ в единциах $\mu_\text{я}$. Относительная погрешность такой величины бодет совпадать с относительной прогрешностью измерения $g_\text{ядер водорода}$ :

\[ \mu = 2,850 \pm 0,13 \]
\[\varepsilon(\mu) =  4,4\%\]

\item Далее перейдем к результатам полученными при опыте с \underline{образоцом №1 - резиной}

\item Значение тока в катушке J = 1,8 А.

\item Получен следующий сигнал Ядерного Магнитного Резонанса на на осцилографическом идтикаторе прибора Ш1-9:

\begin{figure}[H]
 \centering
 \caption{сигнал ЯМР на ядрах водорода - образец №1}
 {\includegraphics[width=0.6\linewidth]{photo2}}
\end{figure}

\item Показания частотометра -- значение резонансной частоты $f_0 = 9,9962 \text{ МГц}$

\item Показание датчика Хола на экране прибора Ш1-10 $B_0 = 235 \text{ мТл}$

\item Таким образом значения g-фактора ядер водорода в данном эскиперменте, найденное по формуле ($\ref{gform}$) составляет:

\[g_\text{ядер водорода} = 5,581 \]

\item Рассчитаем ошибку данного измеения по формуле ($\ref{sigma_g}$):

\[\sigma_g \approx 0,25\]

\item Таким образом:

\[g = 5,581 \pm 0,024 \]
\[\varepsilon(g) =  4,3\%\]

\item Найдем величину $\mu$ в единциах $\mu_\text{я}$  по  формуле ($\ref{mu}$)  при условии, что  I = $\dfrac{1}{2}$. Относительная погрешность такой величины будет совпадать с относительной прогрешностью измерения $g_\text{ядер водорода}$ :

\[ \mu = 2,791 \pm 0,12 \]
\[\varepsilon(\mu) =  4,3\%\]

\end{enumerate}




\subsection*{Ядерный магнтный резонанс на ядрах фтора}

\begin{enumerate}

\item Результаты, полученые при опыте с \underline{образоцом №2 - тефлон}.

\item Значение тока в катушке J = 1,94 А.

\item Получен следующий сигнал Ядерного Магнитного Резонанса на на осцилографическом идтикаторе прибора Ш1-9:

\begin{figure}[H]
 \centering
 \caption{сигнал ЯМР на ядрах фтора - образец №2}
 {\includegraphics[width=0.6\linewidth]{photo3}}
\end{figure}

\item Показания частотометра -- значение резонансной частоты $f_0 = 9,99470 \text{ МГц}$

\item Показание датчика Хола на экране прибора Ш1-10 $B_0 = 247 \text{ мТл}$

\item Таким образом значения g-фактора ядер фтора в данном эскиперменте, найденное по формуле ($\ref{gform}$) составляет:

\[g_\text{ядер водорода} = 5,314 \]

\item Рассчитаем ошибку данного измеения по формуле ($\ref{sigma_g}$):

\[\sigma_g \approx 0,21\]

\item Таким образом:

\[g = 5,31 \pm 0,21 \]
\[\varepsilon(g) =  4,3\%\]

\item Найдем величину $\mu$ в единциах $\mu_\text{я}$  по  формуле ($\ref{mu}$)  при условии, что  I = $\dfrac{1}{2}$. Относительная погрешность такой величины будет совпадать с относительной прогрешностью измерения $g_\text{ядер водорода}$ :

\[ \mu = 2,662 \pm 0,11 \]
\[\varepsilon(\mu) =  4,3\%\]



\end{enumerate}


\subsection*{Ядерный магнтный резонанс на ядрах дейтерия}

\begin{enumerate}

\item Результаты, полученые при опыте с \underline{образоцом №5 - тяжелая вода($D_2O$)}.

\item Ток в катушке электромагнита, установленный в соотвествии с калибровачным граффиком зависимости магнитного поля от тока в кашутках, соотвествующий диапозону значений $B\in[525;545] \text{мТл}$ --  J = 4,63 А.

\item К сожалению шумы в данном эксперименте были настолько велики, что различить сигнал на фото не представляется возможным.

\item Показания частотометра -- значение резонансной частоты $f_0 = 3,4991 \text{ МГц}$

\item Показание датчика Хола на экране прибора Ш1-10 $B_0 = 534 \text{ мТл}$

\item Таким образом значения g-фактора ядер дейьерия в данном эскиперменте, найденное по формуле ($\ref{gform}$) составляет:

\[g_\text{ядер водорода} = 0,860 \]

\item Рассчитаем ошибку данного измеения по формуле ($\ref{sigma_g}$):

\[\sigma_g \approx 0,016\]

\item Таким образом:

\[g = 0,860 \pm 0,016 \]
\[\varepsilon(g) =  1,9\%\]

\item Найдем величину $\mu$ в единциах $\mu_\text{я}$  по  формуле ($\ref{mu}$)  при условии, что  I = 1. Относительная погрешность такой величины будет совпадать с относительной прогрешностью измерения $g_\text{ядер водорода}$ :

\[\mu = 0,860 \pm 0,016 \]
\[\varepsilon(g) =  1,9\%\]


\end{enumerate}
 
\subsection*{Сопоставление полученных данных с табличными значениями} 

\begin{enumerate}

\item Выпишем полученные результаты с приведенными рядом табличными значениями промеренных величины:

\begin{table} [H]
	\centering
	\caption{Таблица результатов} 
\begin{tabular}{|l||c|c|c|c|c|}
	\hline
Образец № & $I$ & $g_\text{я}$ & $g_\text{я, табл.}$ & $\mu$ $(\text{в} ~\mu_\text{я})$ & $ \mu_\text{табл.}, \mu_\text{я} $ \\
\hline
1. Ядро водорода (вода)  & 0.5 & 5.694 $\pm$ 0,25 & 5.58 & 2.850 $\pm$ 0,13  & 2.79  \\
\hline
2. Ядро водорода (резина)   & 0.5 & 5.581 $\pm$ 0,24 & 5.58 & 2.791 $\pm$ 0,12 & 2.79  \\
\hline
3. Ядро фтора (тефлон)& 0.5 & 5.31 $\pm$ 0,21 & 5.26 & 2.66 $\pm$ 0,11 & 2.63  \\
\hline
5. Ядро дейтерия ($D_2$O)  & 1 & 0.860 $\pm$ 0,016 & 0.86 & 0.86 $\pm$ 0,016 & 0.86  \\ 
\hline
\end{tabular}
\end{table}

\end{enumerate}

\subsection*{Величина вклада состояния $^3 D_1$ в магнитный момент дейтрона}

\hspace{7mm} Учитывая тот факт, что полученный нами значения магнитного момента протона с учетом погрешности совпали с табличными, в данном вычислении будем использовать табличные величины:

\[\mu_p = 2,792763 \mu_\text{я}\]
\[\mu_n = -1,91315 \mu_\text{я}\]

\hspace{7mm}Предполагая, что основное состояние дейтрона является смесь двух орбитальных состояний: $^3 S_2$ и $^3 D_1$, вычислим величину вклада $P_D$ по следующей формуле:

\[\mu_d = \mu_n + \mu_p - \frac{3}{2}\left(\mu_n + \mu_p - \frac{1}{2}\right)P_D \]
\[ \Downarrow \]
\[P_D =\dfrac{2}{3} \cdot \dfrac{\mu_n + \mu_p - \mu_d}{\mu_n + \mu_p - \frac{1}{2}}\]

\hspace{7mm} Где будем использовать значение $\mu_d$ полученное экспериментально в предыдущем опыте: 

\[P_D = 0,034\]

\section*{Вывод}

Полученные значения сошлись с табличными в пределах погрешностей.
Теория Ядерного Магнитного Резонанса описанная в Лабораторного практикума применима в иследуемых материалах. 

\end{document}