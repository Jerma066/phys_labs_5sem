\documentclass{letnab}

\begin{document}
\include{cover}

\paragraph{Цель работы:} Определить энергию возбуждения первого уровня атома гелия двумя способами: статическим и динаимческим. Сравнить результаты измерений.


\section*{Теоритический аспект}  

\hspace{7mm}Одним из простых опытов, подтверждающих существование дискретных уровней энергии атомов, является эксперимент, известный под названием опыта Франка и Герца. Схема опыта изображена на рис. \ref{diag} 

\begin{wrapfigure}[10]{r}{7cm}\vspace{-.9cm}
\includegraphics[width=.97\linewidth]{diagram}
\caption{Схема опыта Франка и Герца}
\label{diag}
\end{wrapfigure} 

\hspace{7mm}Разреженный одноатомный газ (в нашем случае - гелий)заполняет трехэлектронную лампу. Электроны, испускаемые разогретым катодом ускоряются в постоянном электрическом поле, созданным между катодом и сетчатым анодом лампы. Передвигаясь от катода к аноду, электроны сталкиваются с атомами гелия. Если энергия электрона, налетающего на атом, недостаточна для того, чтобы перевести его в возбужденное состояние (или ионизовать), то возможны только упругие соударения, при которых электроны почти не теряют энергии, так как их масса в тысячи раз меньше массы атомов.

\hspace{7mm}По мере увеличения разности разности потенциалов между анодом и катодом энергия желектронов увеличивается и, в конце концов,  оказывается достаточной для возбуждения атомов. При таких - неупругих - столкновениях кинетическая энергия налетающего электрона передается одному из атомных электронов, вызывая его переход на свободный энергетический уровень (возбуждение) или совсем отрывая его от атома (ионизация). 

\hspace{7mm}Третьим электродом лампы является коллектор. Между ним и анодом поддерживается небольшое задерживающее напряжение (потенциал коллектора меньше потенциала анода). Ток коллектора пропорциональный числу измеряется микроамперметром.

\begin{wrapfigure}[10]{l}{7cm}\vspace{-.8cm}
\includegraphics[width=.97\linewidth]{dependence}
\caption{Схематический вид зависимости тока коллектора от напряжения на аноде}
\label{dep}
\end{wrapfigure} 

\hspace{7mm}При увеличении потенциала анода ток в лампе вначале растет, подобно тому как это происходит в вакуумном диоде (рис. \ref{dep}). Однако, когда энергия электронов становится достаточной для возбуждения атомов, ток коллектора резко уменьшается. Это происходит потому, что при неупругих
соударениях с атомами электроны почти полностью теряют
свою энергию и не могут преодолеть задерживающего потенциала (около 1 В) между анодом и коллектором. При дальнейшем
увеличении потенциала анода ток коллектора вновь возрастает: электроны,
испытавшие неупругие соударения, при дальнейшем движении к аноду
успевают набрать энергию, достаточную для преодоления задерживающего потенциала.

\hspace{7mm}Следующее замедление роста тока происходит в момент, когда часть электронов
неупруго сталкивается с атомами два раза: первый раз посередине пути,
второй  у анода, и т.д. Таким образом, на кривой зависимости тока коллектора
от напряжения анода имеется ряд максимумов и минимумов, отстоящих
друг от друга на равные расстояния  $\bigtriangleup V$; эти расстояния равны энергии первого
возбужденного состояния (рис.ref{dep}).


\section*{Экспериментальная установка}

\begin{figure}[H]
\centering
\includegraphics[width = 0.9 \tw]{scheme}
\caption{Схема экспериментальной установки}
\label{sch}
\end{figure}

\hspace{7mm}Схема экспериментальной установки изображена
на рис. \ref{sch}. Для опыта используется серийная лампа ионизационного манометра ЛМ-2, заполненная гелием до давления $\simeq1$ Торр. Источником электронов является вольрамовый катод, нагреваемый переменным током. Напряжение накала подаётся от стабилизируемого источника питания Б7-4. Ток накала контролируется амперметром А. Источник Б7-4 включается в цепь тумблером $K_1$.

\hspace{7mm} В качестве анода используется двойная спираль, окружающая катод. Роль коллектора играет полый металлический цилиндр, соосный с катодом и анодом.

\hspace{7mm} Ускоряющее напряжение подаётся на анод от выпрямителя В. Величина этого напряжения регулируется потенциометром $П_3$ и измеряется вольтметром $V_1$. Источник задерживающего напряжения батарея 4,5 В; величина напряжения регулируется потенциометром $П_2$ и измеряется
вольтметром $V_2$. Ток в цепи коллектора регистрируется микроамперметром.

\hspace{7mm} Схему можно переключать из статического режима измерений в динамический режим с помощью ключа $K_3$. На рис. 3 две части сдвоенного ключа $K_3$ изображены отдельно. При динамическом режиме работы ускоряющий потенциал подаётся с понижающего трансорматора T (220/50 В), а ток коллектора регистрируется осциллограом, подключјнным к нагрузочному резистору R. Осцилограф следует синхронизовать от сети 50 Гц.

\hspace{7mm} При определении энергии электронов по разности потенциалов между анодом и катодом следует иметь в виду, что из-за контактной разности потенциалов между катодом и анодом первый максимум не соответствует потенциалу первого возбуждённого уровня. Однако контактная разность потенциалов сдвигает все максимумы одинаково, так что расстояние между ними не меняется.


\section*{Ход работы}

\subsection*{Подготовка приборов к работе}

\subsection*{Получение вольт-амперной характеристики $I_k$ = f($V_a$)на экране осциллографа C1-83.}

\begin{enumerate}

\item Перемещая сигнал ручками и меняя чувствительность канала Y, мы
добились размещения картин в центре экрана. 
\item Меняя значения запирающего напряжения и цену деления канала Y мы проследили за ходом вольт-амперной характеристики на экране ЭО. 
\item Запишем полученный данные в таблицу 1 и продемнстрируем увиденные нами на осцилографе картины:


\newpage

\begin{table}
\centering
\caption{Параметры, при которых производилось наблюдение}
\begin{tabular}{|c|c|c|}
\hline
Задерж. нап., В & Y, мВ/дел. & X, В/дел. \\
\hline
1               & 50         & 25/11     \\
\hline
2               & 50         & 25/11     \\
\hline
3               & 20         & 25/11     \\
\hline
4               & 20         & 25/11     \\
\hline
5               & 20         & 25/11     \\
\hline
6               & 20         & 25/11     \\
\hline
7               & 10         & 25/11     \\
\hline
8               & 10         & 25/11     \\
\hline
9               & 10         & 25/11     \\
\hline
10              & 10         & 25/11     \\
\hline
\end{tabular}

\end{table}

\begin{figure}[H]
\caption{зависимость осцилограммы от запирающего напряжения.}
\begin{minipage}[h]{0.47\linewidth}
\center{\includegraphics[width=1\linewidth]{0}} Зад. нап. 0В \\
\end{minipage}
\hfill
\begin{minipage}[h]{0.47\linewidth}
\center{\includegraphics[width=1\linewidth]{1}} Зад. нап. 1В \\
\end{minipage}
\vfill
\begin{minipage}[h]{0.47\linewidth}
\center{\includegraphics[width=1\linewidth]{2}} Зад. нап. 2В \\
\end{minipage}
\hfill
\begin{minipage}[h]{0.47\linewidth}
\center{\includegraphics[width=1\linewidth]{3}} Зад. нап. 3В \\
\end{minipage}
\end{figure}

\newpage

\begin{figure}[H]

\begin{minipage}[h]{0.47\linewidth}
\center{\includegraphics[width=1\linewidth]{4}} Зад. нап. 4В \\
\end{minipage}
\hfill
\begin{minipage}[h]{0.47\linewidth}
\center{\includegraphics[width=1\linewidth]{5}} Зад. нап. 5В \\
\end{minipage}
\vfill
\begin{minipage}[h]{0.47\linewidth}
\center{\includegraphics[width=1\linewidth]{6}} Зад. нап. 6В \\
\end{minipage}
\hfill
\begin{minipage}[h]{0.47\linewidth}
\center{\includegraphics[width=1\linewidth]{7}} Зад. нап. 7В \\
\end{minipage}
\vfill
\begin{minipage}[h]{0.47\linewidth}
\center{\includegraphics[width=1\linewidth]{8}} Зад. нап. 8В \\
\end{minipage}
\hfill
\begin{minipage}[h]{0.47\linewidth}
\center{\includegraphics[width=1\linewidth]{9}} Зад. нап. 9В \\
\end{minipage}
\end{figure}

\item Из полученных на экране картинок и выбранной по оси X цены деления найдем расстояние между соседними максимумами для трех значений задержвающего напряжения: 4, 6 и 8В. Запишем их в таблицу 2:

\newpage

\begin{table}[]
\centering
\caption{Расстояние между пиками}
\begin{tabular}{|c|c|c|}
\hline
Задерж нап. & $\triangle$V, дел  & $\triangle$V, В \\ \hline
4В          & 6       & 13,64 \\ \hline
6В          & 6       & 13,64 \\ \hline
8В          & 6       & 13,64 \\ \hline
\end{tabular}
\label{pic}
\end{table}

\item Таким образом по полученным данным энергию возбужденя первого уровня атома:

\[ A = 13,64\text{ эВ} \]

\end{enumerate}

\subsection*{Получение вольт-амперной характеристики $I_k$ = f($V_a$)
в статическом режиме измерений}

\begin{enumerate}



\item Снимем зависимость коллекторного тока от анодного напряжения \textbf {$I_k$ = f(Vа)} 
для 3-х различных значений задерживающего напряжения V2 = 4, 6, 8 В. Особенно тщательно (медленно) проводем измерения в тех областях характеристики, где наблюдаются максимумы и минимумы тока $I_k$.


Снятые данные данные занесем в таблицы 3-5:

\begin{table}[H]
\centering
\caption{$V_{зад}$ = 4В}
\begin{tabular}{|c|c|c|c|}
\hline
№  & Ik, дел. & Ik, мА & Va, В \\ \hline
1  & 10       & 50     & 4,23  \\ \hline
2  & 20       & 100    & 7,51  \\ \hline
3  & 30       & 150    & 10,81 \\ \hline
4  & 40       & 200    & 14,5  \\ \hline
5  & 50       & 250    & 18,3  \\ \hline
6  & 53       & 265    & 23    \\ \hline
7  & 53       & 265    & 23,37 \\ \hline
8  & 45       & 225    & 24,65 \\ \hline
9  & 55       & 275    & 28,41 \\ \hline
10 & 65       & 325    & 31,43 \\ \hline
11 & 75       & 375    & 34,47 \\ \hline
12 & 80       & 400    & 38,2  \\ \hline
13 & 78       & 390    & 40,13 \\ \hline
14 & 75       & 375    & 43,13 \\ \hline
15 & 80       & 400    & 50,55 \\ \hline
16 & 90       & 450    & 57,07 \\ \hline
\end{tabular}
\end{table}

\newpage

\begin{table}[]
\centering
\caption{$V_{зад}$ = 6В}
\begin{tabular}{|c|c|c|c|}
\hline
№  & Ik, дел. & Ik, мА & Va, В \\ \hline
1  & 0        & 0      & 0,03  \\ \hline
2  & 20       & 100    & 12,87 \\ \hline
3  & 40       & 200    & 19,64 \\ \hline
4  & 60       & 300    & 21,04 \\ \hline
5  & 70       & 350    & 22,78 \\ \hline
6  & 71       & 355    & 22,88 \\ \hline
7  & 43       & 215    & 25,2  \\ \hline
8  & 50       & 250    & 27,75 \\ \hline
9  & 70       & 350    & 31,4  \\ \hline
10 & 90       & 450    & 36,61 \\ \hline
11 & 91       & 455    & 38,26 \\ \hline
12 & 90       & 450    & 39,23 \\ \hline
13 & 77       & 385    & 46,81 \\ \hline
14 & 90       & 450    & 56,76 \\ \hline
\end{tabular}
\end{table}


\begin{table}[H]
\centering
\caption{$V_{зад}$ = 6В}
\begin{tabular}{|c|c|c|c|}
\hline
№  & Ik, дел. & Ik, мА & Va, В \\ \hline
1  & 0        & 0      & 0,03  \\ \hline
2  & 40       & 200    & 12,87 \\ \hline
3  & 80       & 400    & 19,64 \\ \hline
4  & 95       & 475    & 22,78 \\ \hline
5  & 100      & 500    & 24,82 \\ \hline
6  & 39       & 195    & 26,73 \\ \hline
7  & 60       & 300    & 30,28 \\ \hline
8  & 80       & 400    & 32,75 \\ \hline
9  & 105      & 525    & 38,55 \\ \hline
10 & 90       & 450    & 44,04 \\ \hline
11 & 77       & 385    & 50,68 \\ \hline
12 & 89       & 445    & 70,37 \\ \hline
\end{tabular}
\end{table}

\newpage

\item По данным этих таблиц отметим точки на граффике:

\begin{figure}[h!]
\begin{center}
\includegraphics[width=1\textwidth]{2_1}
\end{center}
\caption{Зависимость $I_k$ = f($V_a$)} \label{gr}
\end{figure}

\item Из графиков $\ref{gr}$ получим значения энергии возбуждения первого уровня атома.

\begin{table}[H]
\centering
\caption{Значение энергии возбуждения}
\begin{tabular}{|l|c|c|c|}
\hline
Vзад., В & 4     & 6    & 8    \\ \hline
A, эВ    & 15,34 & 16,2 & 14,8 \\ \hline
\end{tabular}
\end{table}

\end{enumerate}

\subsection*{Оценка ошибок измерения}
\begin{enumerate}

\item Оценка ошибки динамического метода:

Возьмем ошибку человеческого глаза - половину цены деления, которая была указана в нашем опыте и равняется 25/11 В.

Таким образом измеряя расстояние между пиками мы совершаем ошибку вдвое большую чем огговоренная.

Ошибка в определении энергии возбуждения первого уровня атома равна:

\[ \varrho = 2,3\text{ эВ} \]

А результат первого ксперимента равен:

\[ \varepsilon_d = 13,64 \pm 2,30\text{ эВ}\]  

\item Оценка ошибки статического метода: 

Проведя опыт, мы получили три примерно равных но несколько отличающихся значений. За результат эксперимента будем считать их среднее арифмеическое. Тогда ошибкой будет среднеорифметическое отклонений полученных значений от результата. 

\[ \varepsilon_s = 15,45 \pm 0,50\text{ эВ}\] 


\end{enumerate}

\section{Вывод}  
В ходе опыта были получены значения первого энергетического уровня атома гелия в динамическом ($13.64 \pm 2.3 \text{эВ}$) и статическом ($15.45 \pm 0.50\text{эВ}$) режимах. Полученные результаты совпадают между собой с учетом прогрешности, однако с табличным значением ($19.82 \text{эВ}$) они совпадают лишь по порядку и лежать в одном десятке.

\end{document}


\end{document}
